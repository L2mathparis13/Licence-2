\documentclass{article}
\usepackage{calc}
\usepackage[utf8]{inputenc}
\usepackage{amsmath}
\usepackage{amsthm}
\usepackage{amssymb}
\usepackage{amsfonts}
\usepackage[francais]{babel}
\usepackage{fancyvrb}
\usepackage{hyperref}
\usepackage{mathtools}
\usepackage{tikz,tkz-tab} 
\usepackage[T1]{fontenc}  
\usetikzlibrary{patterns}
\usepackage{multicol}
\usepackage{extarrows}
\usepackage[left=2cm,right=2cm,top=2cm,bottom=2cm]{geometry}
\title{Espaces Vectoriels Normés}
\date{Lundi 15 janvier 2018}
\author{Mise à jour du cours du 05/02}

\begin {document}
\newcommand{\tendplusinf}{\text{$\xlongrightarrow[n \rightarrow  + \infty]{\text{ }}$ }}
\newcommand{\xtendplusinf}{\text{$\xlongrightarrow[x \rightarrow  + \infty]{\text{ }}$ }}
\newcommand{\liminfty}{$\underset{n \rightarrow  + \infty}{\text{lim}}$}


\newcommand{\Bbarre}{$\overline{B}$}
\newcommand{\Bouleun}{\begin{tikzpicture}[scale=0.65]
        \draw (0,0) node[above]{\footnotesize{$a$}};
        \draw (0,0) node{\tiny{$\bullet$}};
        \draw[dashed, color=gray!80, pattern=north west lines, pattern color=blue] (0,0) circle (1);
        \draw (0,0) -- (1,0);
        \draw (1,0) node{\tiny{$\bullet$}};
     \draw (0.5,0) node[above]{\footnotesize{$r$}};
    \end{tikzpicture}
}

\newcommand{\Bouledeux}{
\begin{tikzpicture}[scale=0.65]
    \draw (0,0) node[above]{\footnotesize{$a$}};
    \draw (0,0) node{\tiny{$\bullet$}};
    \draw[blue, pattern=north west lines, pattern color=blue] (0,0) circle (1);
    \draw (0,0) -- (1,0);
    \draw (1,0) node{\tiny{$\bullet$}};
    \draw (0.5,0) node[above]{\footnotesize{$r$}};
\end{tikzpicture}
}


\newcommand{\Bouletrois}{
\begin{tikzpicture}[scale=0.65]
    \draw [->] (0,-1.5) -- (0,1.5);
    \draw [->] (-1.5,0) -- (1.5,0);
    \draw (0,1.65) node{\tiny{$x_2$}};
    \draw (1.8,0) node{\tiny{$x_1$}};
    \draw (0.2,1.2) node{\tiny{$1$}};
    \draw (1.2,0.2) node{\tiny{$1$}};
    \draw (-1.3,0.2) node{\tiny{$-1$}};
    \draw (0.3,-1.2) node{\tiny{$-1$}};
    \draw [dashed, color=gray!80, pattern=north west lines, pattern color=blue] (-1,1) -- (1,1) -- (1,-1) -- (-1,-1) -- cycle;
\end{tikzpicture}
}


\newcommand{\Boulequatre}{
\begin{tikzpicture}[scale=0.65]
    \draw [->] (0,-1.5) -- (0,1.5);
    \draw [->] (-1.5,0) -- (1.5,0);
    \draw (0,1.65) node{\tiny{$x_2$}};
    \draw (1.8,0) node{\tiny{$x_1$}};
    \draw (0.2,1.2) node{\tiny{$1$}};
    \draw (1.2,0.2) node{\tiny{$1$}};
    \draw (-1.3,0.2) node{\tiny{$-1$}};
    \draw (0.3,-1.2) node{\tiny{$-1$}};
    \draw (0,0) node[below left]{\footnotesize{$a$}};
    \draw (0,0) node{\tiny{$\bullet$}};
    \draw[dashed, color=gray!80, pattern=north west lines, pattern color=blue] (0,0) circle (1);
    \draw (0,0) -- (1,0);
    \draw (1,0) node{\tiny{$\bullet$}};
    \draw (0.5,0) node[above]{\footnotesize{$r$}};
\end{tikzpicture}
}

\newcommand{\Boulecinq}{
\begin{tikzpicture}[scale=0.65]
    \draw [->] (0,-1.5) -- (0,1.5);
    \draw [->] (-1.5,0) -- (1.5,0);
    \draw (0,1.65) node{\tiny{$x_2$}};
    \draw (1.8,0) node{\tiny{$x_1$}};
    \draw (0.2,1.2) node{\tiny{$1$}};
    \draw (1.2,0.2) node{\tiny{$1$}};
    \draw (-1.3,0.2) node{\tiny{$-1$}};
    \draw (0.3,-1.2) node{\tiny{$-1$}};
    \draw [dashed, color=gray!80, pattern=north west lines, pattern color=blue] (-1,0) -- (0,1) -- (1,0) -- (0,-1) -- cycle;
\end{tikzpicture}
}
\maketitle
\bigbreak

\parindent=0cm 
Soit $E$ un espace vectoriel.
\smallbreak
Le but de cette partie est de définir la notion de distance entre deux vecteurs de $E$.
\smallbreak
\underline{Ex:} $-$ $E = \mathbb{R}$ :
On peut définir la distance entre deux réels $a$ et $a'$ par $d(a,a') = |a - a'|$.
\smallbreak
\hspace*{0.72cm}$-$ $E = \mathbb{C}$ : On peut poser $z$, $z'$ $\in$ $\mathbb{C}$, $d(z,z') = |z - z'|$. Si on écrit $z = x + iy$ et $z' = x' + iy'$, alors \smallbreak $d(z,z') = |x$ $-$ $x'$ $+$ $i(y$ $-$ $y')| = \sqrt{(x - x')^2 + (y - y')^2}$.
\smallbreak
\hspace*{0.72cm}$-$ $E = \mathbb{R}^2$ : Si ($x$, $y$), ($x'$, $y'$) $\in$ $\mathbb{R}^2$, on peut définir $d((x$, $y$), ($x'$, $y'$))$ = \sqrt{(x - x')^2 + (y - y')^2}$.
\medbreak
\section{Normes et distances sur un espace vectoriel}
\medbreak
\fbox{
	\begin{minipage}{0.9\textwidth}
		\underline{Définition :} Soit $E$ un espace vectoriel sur $\mathbb{R}$. Une norme sur $E$ est par définition une application \smallbreak $N$ : $E$ $\rightarrow$ $\mathbb{R}$, $x$ $\mapsto$ $N(x)$ vérifiant : \smallbreak 
		$-$ $\forall$ $x \in E$,  $N(x)$ $\geqslant$ $0$, et ($N(x)$ $=$ $0$ $\Longleftrightarrow$ $x = 0$), \smallbreak
		$-$ $\forall$ $x \in E$,  $\forall$ $\lambda \in \mathbb{R}$, $N(\lambda x)$ $=$ $|\lambda|N(x)$, \smallbreak
		$-$ (Inégalité triangulaire) $\forall$ $x$, $y \in E$, $N(x$ $+$ $y)$ $\leqslant$ $N(x)$ $+$ $N(y)$.
	\end{minipage}
}
\smallbreak
\fbox{
	\begin{minipage}{0.9\linewidth}
		\underline{Définition :} Un espace vectoriel normé (e.v.n) est un couple ($E$, $N$) où $E$ est un e.v et $N$ une norme sur $E$.
	\end{minipage}
}
\smallbreak
\underline{Propriétés :}
\parindent=1cm  
\smallbreak
$-$ Soient $x_1$, $x_2$, $x_3$ $\in E$ avec ($E$, $N$) un e.v.n. Alors,\smallbreak $N(x_1$ $+$ $x_2$ $+$ $x_3)$ $=$ $N((x_1$ $+$ $x_2$) $+$ $x_3)$ $\leqslant$ $N(x_1$ $+$ $x_2$) $+$ $N(x_3)$ $\leqslant$ $N(x_1)$ $+$ $ N(x_2)$ $+$ $N(x_3)$
\smallbreak
$-$ Si $x_1$, ..., $x_p$ $\in$ $E$, $N(x_1$, ..., $x_p)$ $\leqslant$ $N(x_1)$ $+$ ... $+$ $N(x_p)$.
\smallbreak
$-$ Si $x$, $y$ $\in$ $E$, |$N(x)$ $-$ $N(y)$| $\leqslant$ $N(x$ $-$ $y)$.

\parindent=0cm
\smallbreak
\underline{Démo :} 
\smallbreak
\parindent=1cm
$N(x)$ $=$ $N((x$ $-$ $y)$ $+$ $y)$ $\leqslant$ $N(x$ $-$ $y)$ $+$ $N(y)$, $N(x)$ $-$ $N(y)$ $\leqslant$ $N(x$ $-$ $y)$. \smallbreak Aussi $N(y)$ $-$ $N(x)$ $\leqslant$ $N(y$ $-$ $x)$ $=$ $N((-1)(x$ $-$ $y))$ $=$ |$-$1|$N(x$ $-$ $y)$ $=$ $N(x$ $-$ $y)$. \smallbreak Finalement, |$N(x)$ $-$ $N(y)$| = Max$(N(x)$ $-$ $N(y)$, $N(y)$ $-$ $N(x))$ $\leqslant$ $N(x$ $-$ $y)$.
\parindent=0cm
\smallbreak

\underline{Ex :}
\parindent=1cm
\smallbreak
$-$ $E = \mathbb{R}$ : Posons $N(x)$ = |$x$| (Valeur absolue).\smallbreak $N$ est une norme sur $\mathbb{R}$, car |$x$| $\geqslant$ $0$, (|$x$| $=$ $0$ $\Longleftrightarrow$ $x = 0$), \smallbreak $\forall$ $\lambda \in \mathbb{R}$, |$\lambda x$| $=$ |$\lambda$||$x$|, et $\forall$ $x$, $y \in E$, |$x$ $+$ $y$| $\leqslant$ |$x$| $+$ |$y$|.
\smallbreak
$-$ $E = \mathbb{C}$ : Posons $N(x)$ = |$x$| (Module).\smallbreak $N$ est une norme sur $\mathbb{C}$, car |$z$| $\geqslant$ $0$, (|$z$| $=$ $0$ $\Longleftrightarrow$ $z = 0$), \smallbreak $\forall$ $\lambda \in \mathbb{C}$, |$\lambda z$| $=$ |$\lambda$||$z$|, et $\forall$ $z$, $z' \in E$, |$z$ $+$ $z'$| $\leqslant$ |$z$| $+$ |$z'$|.
\smallbreak
$-$ Espaces euclidiens :
\smallbreak Soit $E$ un e.v. Une forme bilinéaire est une application $B$ : $E$ $\times$ $E$ $\rightarrow$ $\mathbb{R}$, $(x$, $y)$ $\mapsto$ $B(x$, $y)$ telle qu'elle est \smallbreak linéaire en chacune de ses variables. \smallbreak Un produit scalaire sur un e.v $E$ est une forme bilinéaire symétrique sur $E$ définie positive au sens suivant : \smallbreak $\forall$ $x \in E$,  $B(x,x)$ $\geqslant$ $0$, et $B(x,x)$ $=$ $0$ $\Longleftrightarrow$ $x = 0$. Un espace euclidien est un e.v muni d'un produit scalaire. \smallbreak On pose $B(x$, $y)$ = $\overset{n}{\underset{j = 1}{\sum}} x_j y_j$. $B$ est un produit scalaire sur $\mathbb{R}^n$ avec ($x$, $y$ $\in$ $\mathbb{R}^n$).
\smallbreak
On pose maintenant : $N(x)$ $=$ $\sqrt{B(x, x)}$. Alors, $N$ est une norme sur $E$.\smallbreak
\underline{Démo :} \parindent=2cm \smallbreak
$i$/ Par définition, $N(x)$ $\geqslant$ $0$, et $N(x)$ $=$ $0$ $\Longleftrightarrow$ $B(x,x)$ $=$ $0$ $\Longleftrightarrow$ $x = 0$ \smallbreak
$ii$/ Si $\lambda$ $\in$ $\mathbb{R}$, $x$ $\in$ $E$, $N(\lambda x)$ $=$ $\sqrt{B(\lambda x,\lambda x)} $ $=$ $\sqrt{\lambda^2 B(x,x)}$ $=$ |$\lambda$|$\sqrt{B(x,x)}$ $=$ $|\lambda|N(x)$,\smallbreak
$iii$/ On va utiliser le lemme : $\forall$ $x$, $y \in E$, posons $p(\lambda)$ $=$ $B(x + \lambda y, x)$ $+$ $\lambda B(x + \lambda y, y)$. Si on pose \smallbreak $z$ $=$ $x$ $+$ $\lambda y$ fixé, et $u(\omega)$ $=$ $B(z, \omega)$, on a écrit que ($u(x$ $+$ $\lambda y)$ $=$ $u(x)$ $+$ $\lambda u(y)$). \smallbreak On aura $p(\lambda)$ $=$ $B(x$, $x)$ $+$ $\lambda B(y$, $x)$ $+$ $\lambda B(x$, $y)$ $+$ $\lambda^2B(y$, $y)$ donc $p(\lambda)$ $=$ $B(x$, $x)$ $+$ $2\lambda B(y$, $x)$ $+$\smallbreak $\lambda^2B(y$, $y)$ donc $\lambda$ $\mapsto$ $p(\lambda)$ est un polynôme de degré $\leqslant$ $2$ et $p(\lambda)$ $=$ $B(x + \lambda y, x + \lambda y)$ $\geqslant$ $0$. \smallbreak Or, si un polynôme de degré $\leqslant$ $2$ ne change pas de signe, son discriminant est $\leqslant$ $0$. \smallbreak ($2B(x$, $y))^2$ $-$ $4B(x$, $x)B(y$, $y)$ $\leqslant$ $0$, donc |$B(x$, $y)$| $\leqslant$ $\sqrt{B(x, x)}$ $\sqrt{B(y, y)}$. On a donc $N(x$ $+$ $y)^2$ \smallbreak $=$ $B(x$  $+$ $y$, $x$  $+$ $y)$ $=$ $B(x$, $x$  $+$ $y)$ $+$ $B(y$, $x$  $+$ $y)$ $=$ $B(x$, $x)$ $+$ $B(y$, $x)$ $+$ $B(x$, $y)$ $+$ $B(y$, $y)$ $=$\smallbreak $B(x$, $x)$ $+$ $2B(x$, $y)$ $+$ $B(y$, $y)$ $\leqslant$ $N(x)^2$ $+$ $N(x)N(y)$ $+$ $N(y)^2$. \smallbreak On a obtenu $N(x$ $+$ $y)^2$ $\leqslant$ $(N(x)$ $+$ $N(y))^2$ soit $N(x$ $+$ $y)$ $\leqslant$ $N(x)$ $+$ $N(y)$
\parindent=1cm
\smallbreak
$-$ $E = \mathbb{R}^n$ : $B(x$, $y)$ $=$ $\Sigma$ $x_iy_i$. La norme obtenue se note ||..||$_2$ et pour $x$ $=$ 
$
\begin{bmatrix}
	x_1 \\
	\vdots \\
	x_n
\end{bmatrix}
$
 est donnée par \smallbreak ||x||$_2$ $\overset{ def }{=}$ $($ $\overset{n}{\underset{i = 1}{\sum}} x_i^2)^{\frac{1}{2}}$
\smallbreak
$-$ Autre exemple de normes : Soient $E$, $E'$ deux e.v et $\varphi$ : $E$ $\rightarrow$ $E'$ linéaire injective. \smallbreak Soit $N'$ une norme sur $E'$. Pour $x$ $\in$ $E$, posons $N(x)$ $\overset{ def }{=}$ $N'(\varphi (x))$. Alors $N$ est une norme sur $E$:
\smallbreak $i$/ $N(x)$ $\geqslant$ $0$ de plus $N(x)$ $=$ $0$ $\Longrightarrow$ $N'(\varphi(x))$ $=$ $0$ $\Longrightarrow$ $\varphi(x)$ $=$ $0$ (car $N'$ est une norme) $\Longrightarrow$ $x$ $=$ $0$. (puisque \smallbreak$\varphi$ est injective).
\smallbreak
$ii$/ $N(\lambda x) = N'(\varphi(\lambda x))$ $=$ $N'(\lambda \varphi(x))$ $=$ |$\lambda$|$N'(\varphi(x))$ $=$ |$\lambda$|$N(x)$
\smallbreak
$iii$/ $N(x$ $+$ $y)$ $=$ $N'(\varphi(x$ $+$ $y))$ $=$ $N'(\varphi(x)$ $+$ $\varphi(y))$ $\leqslant$ $N'(\varphi(x))$ $+$ $N'(\varphi(y))$ $=$ $N(x)$ $+$ $N(y)$
\parindent=0cm
\subsection{Normes usuelles de $\mathbb{R}^p$}
Soit $x$ $=$ $
\begin{bmatrix}
	x_1 \\
	\vdots \\
	x_p
\end{bmatrix}
$
$\in$ $\mathbb{R}^p$, les normes usuelles de $\mathbb{R}^p$ sont définies par :

\smallbreak
\fbox{
	\begin{minipage}{0.9\textwidth}
		\center 
		||$x$||$_1$ $\overset{ def }{=}$ $\overset{p}{\underset{j = 1}{\sum}}|x_j|$,\hspace*{0.5cm}
		||$x$||$_2$ $\overset{ def }{=}$ $($$\overset{p}{\underset{j = 1}{\sum}} x_j^2)^{\frac{1}{2}}$,\hspace*{0.5cm}
		||$x$||$_{\infty}$ $\overset{ def }{=}$ $\underset{j \in \left\{ 0,\text{ }...,\text{ }p \right\} }{\text{Max}}$[$|x_j|$].\hspace*{0.5cm} 

		\flushleft
	\end{minipage}
}
\smallbreak
\underline{Propriété :} ||..||$_1$, ||..||$_2$, ||..||$_{\infty}$, sont des normes de $\mathbb{R}^p$.
\smallbreak
\underline{Démo :}
\parindent=1cm
\smallbreak
$-$ Pour ||..||$_2$ : voir plus haut.
\smallbreak
$-$ Pour ||..||$_1$ :
\smallbreak
Soit $x$ $\in E$, ||$x$||$_1$ $=$ $\overset{p}{\underset{j = 1}{\sum}}|x_j|$ $\geqslant$ $0$, et $\overset{p}{\underset{j = 1}{\sum}}|x_j|$ $=$ $0$ $\Longleftrightarrow$ $\forall$ $j$ $x_j = 0$ $\Longleftrightarrow$ $x = 0$.
\smallbreak
Soit $\lambda$ $\in$ $\mathbb{R}$, ||$\lambda x$||$_1$ $=$ $\overset{p}{\underset{j = 1}{\sum}}|\lambda x_j|$ $=$ |$\lambda$|$(\overset{p}{\underset{j = 1}{\sum}}|x_j|)$ $=$ |$\lambda$|||$x$||$_1$.
\smallbreak
Soient $x$, $y$ $\in$ $\mathbb{R}^p$, ||$x$ $+$ $y$||$_1$ $=$ $\overset{p}{\underset{j = 1}{\sum}}|x_j + y_j|$ $\leqslant$ $\overset{p}{\underset{j = 1}{\sum}}(|x_j| + |y_j|)$ $=$ $\overset{p}{\underset{j = 1}{\sum}}|x_j|$ $+$ $\overset{p}{\underset{j = 1}{\sum}}|y_j|$ $=$ ||$x$||$_1$ $+$ ||$y$||$_1$
\smallbreak
$-$ Pour ||..||$_{\infty}$ :
\smallbreak
||$x$||$_{\infty}$ $=$ 0 $\Longrightarrow$ $\underset{j \in \left\{ 1,\text{ }...,\text{ }p \right\} }{\text{Max}}$[$|x_j|$] $=$ $0$. $\Longrightarrow$ $\forall$ $j$ $x_j = 0$ $\Longleftrightarrow$ $x = 0$.
\smallbreak
Vérifions l'inégalité triangulaire : |$x_j$ + $y_j$| $\leqslant$ |$x_j$| + |$y_j$| $=$ ||$x_j$||$_{\infty}$ $+$ ||$y_j$||$_{\infty}$ donc \smallbreak $\underset{j \in \left\{ 1,\text{ }...,\text{ }p \right\} }{\text{Max}}$[|$x_j$| + |$y_j$|] $\leqslant$ ||$x_j$||$_{\infty}$ + ||$y_j$||$_{\infty}$.
\smallbreak
\parindent=0cm
\underline{Propriété :} $\forall$ $x$ $\in$ $\mathbb{R}^p$, ||$x$||$_{\infty}$ $\leqslant$ ||$x$||$_2$ $\leqslant$ ||$x$||$_1$ $\leqslant$ $p$||$x$||$_{\infty}$.
\smallbreak
\underline{Démo :} \parindent=1cm $\forall$ $j \in \left\{ 1,\text{ }...,\text{ }p \right\} $, |$x_j$| $\leqslant$ $\left(\overset{p}{\underset{j = 1}{\sum}} x_j^2\right)^{\frac{1}{2}}$ donc ||$x$||$_{\infty}$ $=$ $\underset{j \in \left\{ 1,\text{ }...,\text{ }p \right\} }{\text{Max}}$[$|x_j|$] $\leqslant$ ||$x$||$_2$.
\smallbreak
Pour montrer que ||$x$||$_2$ $\leqslant$ ||$x$||$_1$, il suffit de vérifier que ||$x$||$_2^2$ $\leqslant$ ||$x$||$_1^2$
\smallbreak
Soit $\left( \overset{p}{\underset{j = 1}{\sum}} x_j\right)^2$ $=$ $\left( \overset{p}{\underset{j = 1}{\sum}} x_j\right)$ $\left(\overset{p}{\underset{j = 1}{\sum}} x_j\right)$ $=$ $\overset{p}{\underset{j = 1}{\sum}}\left( \overset{p}{\underset{i = 1}{\sum}} |x_j||x_i|\right)$ $=$ $\overset{p}{\underset{j = 1}{\sum}} x_j^2$ $+$ $\underset{i \neq j}{\overset{p}{\underset{j = 1}{\sum}} \overset{p}{\underset{i = 1}{\sum}}}$ |$x_j$||$x_i$|. Or, $\underset{i \neq j}{\overset{p}{\underset{j = 1}{\sum}} \overset{p}{\underset{i = 1}{\sum}}}$ |$x_j$||$x_i$| $\geqslant$ $0$, donc \smallbreak ||$x$||$_2^2$ $\leqslant$ ||$x$||$_1^2$, enfin ||$x$||$_2$ $\leqslant$ ||$x$||$_1$.
\smallbreak
||$x$||$_1$ $=$ $\overset{p}{\underset{j = 1}{\sum}}|x_j|$ $\leqslant$ $\overset{p}{\underset{j = 1}{\sum}}||x||_{\infty} $ $=$ ||$x$||$_{\infty}$ $\overset{p}{\underset{j = 1}{\sum}} $1 = $p$||$x$||$_{\infty}$
\smallbreak
\parindent=0cm
\underline{Remarque :} \parindent=1cm \smallbreak Si $E$ est un e.v de dimension finie $p$, et si ($e_1$, ..., $e_p$) est une base de $E$, alors $\forall$ $x$ $\in$ $E$ s'écrit de \smallbreak manière unique $x$ $=$ $\overset{p}{\underset{j = 1}{\sum}}x_j e_j$. On peut donc définir des normes $N_1$, $N_2$, $N_{\infty}$ sur $E$ en posant \smallbreak $N_{\ell}$ $\overset{ def }{=}$ $\left| \left|
\begin{bmatrix}
	x_1 \\
	\vdots \\
	x_n
\end{bmatrix}
\right| \right|_{\ell}$, $\ell$ $\in$ $\left\{1, 2, \infty \right\}$

\subsection{Norme produit}
\parindent=0cm
Soient ($E_1$, $N_1$),($E_2$, $N_2$) deux e.v.n. Soit $E_1$ $\times$ $E_2$ $=$ $\left\{ x = (x_1, x_2),\text{ }  x_1 \in E_1,\text{ } x_2 \in E_2 \right\}$. 

\smallbreak
\fbox{
	\begin{minipage}{0.9\textwidth}
		\underline{Définition :} $N(x)$ $\overset{ def }{=}$ Max[$N_1(x_1)$, $N_2(x_2)$] est une norme sur $E$, et est appelée norme produit.
	\end{minipage}
}

\subsection{Distance associée à une norme}

\fbox{
	\begin{minipage}{0.9\textwidth}
		\underline{Définition :} Soit ($E$, $N$) un e.v.n. la distance $d(x$, $y)$ entre $x$ $\in$ $E$ et $y$ $\in$ $E$, associée à $N$ est par définition $d(x$, $y)$ $=$ $N(x$ $-$ $y)$.
	\end{minipage}
}

\smallbreak
\underline{Propriété :} La distance précédente est une application $d$ : $E$ $\times$ $E$ $\rightarrow$ $\mathbb{R}$ vérifiant :
\parindent=1cm
\smallbreak
$i$/ $\forall$ ($x$, $y$) $\in$ $E$ $\times$ $E$ $d(x$, $y)$ $\geqslant$ $0$ et ($d(x$, $y)$ $=$ $0$ $\Longleftrightarrow$ $x = y$) \smallbreak
$ii$/ (symétrie) $\forall$ $x$, $y$ $\in$ $E$, $d(x$, $y)$ $=$ $d(y$, $x)$ \smallbreak
$iii$/ (Inégalité triangulaire) $\forall$ $x$, $y$, $z$ $\in$ $E$, $d(x$, $z)$ $\leqslant$ $d(x$, $y)$ $+$ $d(y$, $z)$ 
\parindent=0cm
\smallbreak
\underline{Démo :} \parindent=1cm \smallbreak $ii$/ $d(x$, $y)$ $=$ $N(y$ $-$ $x)$ $=$ $N((-1)(x$ $-$ $y))$ $=$ |$-$1|$N(x$ $-$ $y)$ $=$ $N(x$ $-$ $y)$ $=$ $d(y$, $x)$ \smallbreak
$iii$/ $d(x$, $z)$ $=$ $N(x$ $-$ $z)$ $=$ $N((x$ $-$ $y)$ $+$ $(y$ $-$ $z))$ $\leqslant$ $N(x$ $-$ $y)$ $+$ $N(y$ $-$ $z)$ $=$ $d(x$, $y)$ $+$ $d(y$, $z)$ 
\parindent=0cm
\smallbreak
\underline{Remarque :} \parindent=1cm \smallbreak De manière générale, si $E$ est une ensemble, on définit une $d$ distance sur $E$ comme une application vérifiant \smallbreak $d$ : $E$ $\times$ $E$ $\rightarrow$ $\mathbb{R}$ vérifiant $i$/, $ii$/, $iii$/.
\parindent=0cm
\smallbreak
C'est une notion de distance plus générale de la distance associée à une norme. Si $d(x$, $y)$ $=$ $N(x$ $-$ $y)$. On peut prendre par exemple pour tout $a$ $\in$ $E$, $d(x$ $+$ $a$, $y$ $+$ $a)$ $=$ $N((x$ $+$ $a)$ $-$ $(y$ $+$ $a))$ $=$ $N(x$ $-$ $y)$ $=$ $d(x$, $y)$. Cette propriété n'est pas toujours vraie pour une distance "normale".
\smallbreak
Soit ($E$, $N$) un e.v.n. Soit $a$ $\in$ $E$.\smallbreak
\fbox{
	\begin{minipage}{0.9\textwidth}
		\underline{Définition :} \smallbreak
		$i$/ Soit $r>0$, la boule ouverte de centre $a$ et de rayon $r$ est par définition $B(a$, $r)$ $=$  $\left\{ x \text{ }\in \text{ }E,\text{ }N(x\text{ }-\text{ }a)\text{ }<\text{ }r \right\}$ $=$  $\left\{ x \text{ }\in \text{ }E,\text{ }d(x,\text{ }a)\text{ }<\text{ }r \right\}$ \smallbreak
		$ii$/ Soit $r\geqslant 0$, la boule fermée de centre $a$ et de rayon $r$ est par définition \Bbarre$(a$, $r)$ $=$ $\left\{ x \text{ }\in \text{ }E,\text{ }N(x\text{ }-\text{ }a)\text{ } \leqslant \text{ }r \right\}$ $=$ $\left\{ x \text{ }\in \text{ }E,\text{ }d(x,\text{ }a)\text{ } \leqslant \text{ }r \right\}$
	\end{minipage}
}
\smallbreak
\underline{Remarque :} \parindent=1cm \smallbreak
$-$ Soit $r>0$, alors $a$ $\in$ $B(a$, $r)$ $\subset$ \Bbarre$(a$, $r)$ \smallbreak
$-$ \Bbarre$(a$, $0)$ $=$ $\left\{ a \right\}$ \smallbreak
$-$ Si $r<r'$ $B(a$, $r)$ $\subset$ $B(a$, $r')$, \Bbarre$(a$, $r)$ $\subset$ \Bbarre$(a$, $r')$

\parindent=0cm
\smallbreak
\underline{Ex :}
\parindent=1cm
\smallbreak
$-$ $E = \mathbb{R}$ , $N(x)$ $=$ |$x$|. Soit $a$ $\in$ $\mathbb{R}$, alors $B(a$, $r)$ $=$ $\left\{ x \text{ }\in \text{ }E,\text{ }|x\text{ }-\text{ }a|\text{ }<\text{ }r \right\}$ $=$ ]$a$ $-$ $r$, $a$ $+$ $r$[,
\setlength{\unitlength}{0.8cm} 
\begin{picture}(3,0.5)(0,0.15) \put(0,0.25){\line(1,0){3}}
\put(0.4,0.16){]}
\put(2.6,0.16){[}
\put(1.5,0.16){|}
\put(0,0.6){\footnotesize{$a$ $-$ $r$}}
\put(1.45,0.6){\footnotesize{$a$}}
\put(2.2,0.6){\footnotesize{$a$ $+$ $r$}}
\put(0.5,0.16){\textcolor{blue}{$\smallsetminus \smallsetminus \smallsetminus \smallsetminus \smallsetminus$}}

\end{picture}
\smallbreak

\Bbarre$(a$, $r)$ $=$ $\left\{ x \text{ }\in \text{ }\mathbb{R},\text{ }|x\text{ }-\text{ }a|\text{ } \leqslant \text{ }r \right\}$ $=$[$a$ $-$ $r$, $a$ $+$ $r$], \begin{picture}(3,0.5)(0,0.15) \put(0,0.25){\line(1,0){3}}
\put(2.6,0.16){]}
\put(0.4,0.16){[}
\put(1.5,0.16){|}
\put(0,0.6){\footnotesize{$a$ $-$ $r$}}
\put(1.45,0.6){\footnotesize{$a$}}
\put(2.2,0.6){\footnotesize{$a$ $+$ $r$}}
\put(0.35,0.16){\textcolor{blue}{$\smallsetminus \smallsetminus \smallsetminus \smallsetminus \smallsetminus \smallsetminus$}}
\end{picture}

\smallbreak


$-$ $E = \mathbb{C}$, $N(z)$ $=$ |$z$|, pour $z$ $\in$ $\mathbb{C}$. Soient $a$, $r>0$.
\begin{multicols}{2}
\parindent=1cm
Alors $B(a$, $r)$ $=$ $\left\{ z \text{ }\in \text{ }\mathbb{C},\text{ }|z\text{ }-\text{ }a|\text{ }<\text{ }r \right\}$ c'est le 
\hspace*{1cm}disque \underline{ouvert} de centre $a$, et de rayon $r$:

\columnbreak
\Bouleun


\end{multicols}
\begin{multicols}{2}
\parindent=1cm
Alors \Bbarre$(a$, $r)$ $=$ $\left\{ z \text{ }\in \text{ }\mathbb{C},\text{ }|z\text{ }-\text{ }a|\text{ } \leqslant \text{ }r \right\}$ c'est le 
\hspace*{1cm}disque \underline{fermé} de centre $a$, et de rayon $r$:

\columnbreak
\Bouledeux



\end{multicols}
$-$ $E = \mathbb{R}^p$ muni des normes ||..||$_1$, ||..||$_2$, ||..||$_{\infty}$, on a vu que $\forall$ $x$ $\in$ $\mathbb{R}^p$, ||$x$||$_{\infty}$ $\leqslant$ ||$x$||$_2$ $\leqslant$ ||$x$||$_1$ $\leqslant$ $p$||$x$||$_{\infty}$\smallbreak Notons $B_{\ell}(a$, $r)$ $=$ $\left\{ x \text{ }\in \text{ }\mathbb{R}^p,\text{ }||x\text{ }-\text{ }a||_{\ell}\text{ } \leqslant \text{ }r \right\}$. $\ell$ $\in$ $\left\{1, 2, \infty \right\}$

\smallbreak 
Soit $r>0$, soit $x$ $\in$ $B_{\ell}(0$, $\frac{r}{p})$. Alors ||$x$||$_{\infty}$ $<$ $\frac{r}{p}$, donc ||$x$||$_1$ $\leqslant$ $p$||$x$||$_{\infty}$ $<$ $p\frac{r}{p}$ $=$ $r$ or $x$ $\in$ $B_1(0$, $r)$.
\smallbreak
On en déduit que $B_{\infty}(0$, $\frac{r}{p})$ $\subset$ $B_1(0$, $r)$
De même $B_1(0$, $r)$ $\subset$ $B_2(0$, $r)$ $\subset$ $B_{\infty}(0$, $r)$.
\bigbreak
\center 
$\forall$ $x$ $\in$ $E$, $B_{\infty}(0$, $\frac{r}{p})$ $\subset$ $B_1(0$, $r)$ $\subset$ $B_2(0$, $r)$ $\subset$ $B_{\infty}(0$, $r)$.

\flushleft
\hspace*{1cm}Cas de $p$ $=$ $2$, $r$ $=$ $1$
\smallbreak

\begin{multicols}{2}

\parindent=1.5cm
Alors $B_{\infty}(0$, $1)$ $=$ $\left\{x=(x_1,\text{ }x_2) \in \mathbb{R}^2,||x||_{\infty} \right\}$ \smallbreak $=$ $\left\{(x_1,\text{ }x_2) \in \text{ }\mathbb{R}^2,\text{Max[|}x_1\text{|, |}x_2\text{|]} <1 \right\}$

\smallbreak
$=$ $\left\{(x_1,\text{ }x_2) \in \mathbb{R}^2,|x_1| < 1 \text{ et } |x_2| < 1 \right\}$
\smallbreak
\hspace{1cm}
\columnbreak

\Bouletrois
\end{multicols}

\smallbreak

\begin{multicols}{2}

\parindent=1.5cm
Alors $B_{2}(0$, $1)$ $=$ $\left\{x=(x_1,\text{ }x_2) \in \mathbb{R}^2,||x||_{2} \right\}$ \smallbreak $=$ $\left\{(x_1,\text{ }x_2) \in \text{ }\mathbb{R}^2,\sqrt{x_{1}^2 + x_{2}^2} <1 \right\}$
\smallbreak
$=$ $\left\{(x_1,\text{ }x_2) \in \mathbb{R}^2,|x_1| < 1 \text{ et } |x_2| < 1 \right\}$ \smallbreak
$=$ $\left\{z = x_1 + ix_2) ,|z| < 1 \right\}$
\columnbreak

\Boulequatre
\end{multicols}
\parindent=1.5cm
\smallbreak
Alors $B_{\infty}(0$, $1)$ $=$ $\left\{x=(x_1,\text{ }x_2) \in \mathbb{R}^2,||(x_1,\text{ }x_2)||_{1} \right\}$ $=$ $\left\{(x_1,\text{ }x_2) \in \mathbb{R}^2,|x_1| + |x_2| < 1 \right\}$
\begin{multicols}{2}

\parindent=1.5cm
Alors $B_{1}(0$, $1)$ $\cap$ $\left\{(x_1,\text{ }x_2), x_1 \leqslant 0, x_2 \right\}$ $=$ $A$. \smallbreak $B_{1}(0$, $1)$ s'obtient par symétrie par A, par \smallbreak rapport aux axes.
\smallbreak
\hspace{1cm}
\smallbreak
\hspace{1cm}
\smallbreak
\columnbreak

\Boulecinq
\end{multicols}
\parindent=0cm
\smallbreak
\fbox{
	\begin{minipage}{0.9\textwidth}
		\underline{Définition :} \smallbreak
		On dit que deux normes $N_1$ et $N_2$ sont équivalentes ssi $\exists$ $C>0$, telle que $\forall$ $x$ $\in$ $E$,
		 $N_1(x)$ $\leqslant$ $CN_2(x)$ et  $N_2(x)$ $\leqslant$ $CN_1(x)$.
	\end{minipage}
}
\smallbreak
\underline{Remarque :} \smallbreak
On définit aussi "$N_1$ et $N_2$ sont équivalentes" par:
\parindent=1cm
\smallbreak
$\exists$ $C_1>0$, telle que $\forall$ $x$ $\in$ $E$, $N_2(x)$ $\leqslant$ $C_1 N_1(x)$ et,
$\exists$ $C_2>0$, telle que $\forall$ $x$ $\in$ $E$, $N_1(x)$ $\leqslant$ $C_2 N_2(x)$.

\parindent=1cm
\smallbreak
Cette définition est équivalente à la précédente. 

Déf 1 $\Longrightarrow$ Déf 2, c'est évident, on prend $C_1$ $=$ $C_2$ $=$ $C$. Déf 1 $\Longrightarrow$ Déf 2, en prennant $C$ $=$ Max[$C_1$, $C_2$].
\parindent=0cm

\underline{Ex :}
Sur $\mathbb{R}^p$ ||..||$_1$, ||..||$_2$, ||..||$_{\infty}$, sont des normes équivalentes (car $\forall$ $x$ $\in$ $\mathbb{R}^p$, ||$x$||$_{\infty}$ $\leqslant$ ||$x$||$_2$ $\leqslant$ ||$x$||$_1$ $\leqslant$ $p$||$x$||$_{\infty}$.)

\section{Limites et continuité}
\fbox{
	\begin{minipage}{0.9\textwidth}
		\underline{Définition :} \smallbreak
		On considère ($E$, $N$) un e.v.n. Soit $(u_n)_{n \in \mathbb{N}}$ une suite d'éléments de $E$. On dit que $u_n$ converge vers $\ell$ ssi la suite réelle ($N(u_n$ $-$ $\ell))_{n \in \mathbb{N}}$ converge vers $0$.
	\end{minipage}
}

\smallbreak
\underline{Propriété :} Supposons que $u_n$ $\tendplusinf$ $\ell$,  $u'_n$ $\tendplusinf$ $\ell'$. Alors,  $u_n$  $+$ $u'_n$ $\tendplusinf$ $\ell$ $+$ $\ell'$.

\smallbreak
\underline{Démo :} 
\parindent=1cm
\smallbreak
$N((u_n$ $+$ $u'_n)$ $-$ $(\ell$ $+$ $\ell'))$ $=$ $N((u_n$ $-$ $\ell)$ $+$ $(u'_n$ $-$ $\ell'))$ $\leqslant$ $\underset{\tendplusinf \text{ }0}{\underbrace{N(u_n\text{ }-\text{ }\ell)}}$ $+$ $\underset{\tendplusinf \text{ }0}{\underbrace{ N(u'_n\text{ }-\text{ }\ell')}}$ 
\smallbreak
donc $N((u_n$ $+$ $u'_n)$ $-$ $(\ell$ $+$ $\ell'))$ $\tendplusinf$ $0$. On en déduit que $u_n$  $+$ $u'_n$ $\tendplusinf$ $\ell$ $+$ $\ell'$
\smallbreak
\parindent=0cm
\underline{Remarque :} \parindent=1cm \smallbreak
Si $(u_n)_{n \in \mathbb{N}}$ une suite de ($E$, $N$) converge, la limite est unique : Supposons que $u_n$ $\longrightarrow$ $\ell$,  $u'_n$ $\longrightarrow$ $\ell'$. Alors, \smallbreak $0$ $\leqslant$ $N(\ell$ $-$ $\ell')$ $=$ $N((\ell$ $-$ $u_n)$ $+$ $(\ell'$ $-$ $u_n))$ $\leqslant$ $N(\ell$ $-$ $u_n)$ $+$ $N(\ell'$ $-$ $u_n)$ $\tendplusinf$ $0$\smallbreak $\Longrightarrow$ $0$ $\leqslant$ $N(\ell$ $-$ $\ell')$ $\leqslant$ \liminfty $\overset{n}{\underset{i = 1}{\sum}}u_i$ $=$ $0$ d'où $N(\ell$ $-$ $\ell')$ $=$ $0$, donc $\ell$ $-$ $\ell'$ $=$ $0$.

\smallbreak
\parindent=0cm
\underline{Proposition :} \parindent=1cm \smallbreak
Soient $N_1$, $N_2$ deux normes equivalentes sur un e.v $E$. Soit $(u_n)_{n \in \mathbb{N}}$ une suite de $E$, soit $\ell$ $\in$ $E$. Alors \smallbreak ($N_1(u_n$ $-$ $\ell)$ $\Longrightarrow$ $0$) $\Longleftrightarrow$ ($N_2(u_n$ $-$ $\ell)$ $\Longrightarrow$ $0$). (($i$) $\Longleftrightarrow$ ($ii$)).

\parindent=0cm
\underline{Démo :} \parindent=1cm \smallbreak Montrons que $i$/ $\Longrightarrow$ $ii$/. Comme $N_1$, $N_2$ sont équivalents, $\exists$ $C>0$, tel que $\forall$ $x$ $\in$ $E$, $N_2(x)$ $\leqslant$ $CN_1(x)$ donc \smallbreak $0$ $\leqslant$ $N_2(u_n$ $-$ $\ell)$ $\leqslant$ $\underset{ \tendplusinf \text{ }0}{\underbrace{CN_1(u_n\text{ }-\text{ }\ell)}}$. Donc $N_2(u_n$ $-$ $\ell)$ $\longrightarrow$ $0$. De même $ii$/ $\Longrightarrow$ $i$/ en inversant $N_1$ et $N_2$.

\parindent=0cm
\underline{Corollaire :} \parindent=1cm \smallbreak
Comme les normes ||..||$_1$, ||..||$_2$, ||..||$_{\infty}$ sont équivalentes : \smallbreak Une suite de $\mathbb{R}^p$ converge vers $\ell$ $\in$ $\mathbb{R}^p$ pour l'une de ces normes ssi elle converge pour une autre.

\parindent=0cm
\underline{Remarque :} \parindent=1cm \smallbreak
Pour montrer qu'une suite $(u_n)_{n \in \mathbb{N}}$ de ($E$, $N$) converge vers $\ell$, il est équivalent de montrer que \smallbreak la suite $(u_n$ $-$ $\ell)_{n \in \mathbb{N}}$ converge vers 0.

\parindent=0cm
\underline{Proposition :} \parindent=1cm \smallbreak
Soit $(u_n)_{n \in \mathbb{N}}$ une suite de $\mathbb{R}^p$, $u_n$ $=$ $
\begin{bmatrix}
	x_{1,n} \\
	\vdots \\
	x_{p,n}
\end{bmatrix}
$ où  ($(x_{j,n})_{n \in \mathbb{N}}$ est une suite de $\mathbb{R}$ $\forall$ $j \in \left\{ 1,\text{ }...,\text{ }p \right\} $). Alors \smallbreak $(u_n)_{n \in \mathbb{N}}$ converge vers $\ell$ $=$ $
\begin{bmatrix}
	\ell_1 \\
	\vdots \\
	\ell_p
\end{bmatrix}
$ $\in \mathbb{R^p}$ muni de l'une des normes ||..||$_1$, ||..||$_2$, ||..||$_{\infty}$ $\Longleftrightarrow$ Les suites $(x_{p,n})_{n \in \mathbb{N}}$ \smallbreak vers $\ell_j$, $\forall$ $j \in \left\{ 1,\text{ }...,\text{ }p \right\} $).

\parindent=0cm
\smallbreak
\underline{Démo :} \parindent=1cm \smallbreak Par la remarque précédente, on peut supposer $\ell$ $=$ $0$. On doit montrer ||$u_n$||$_k$ $\tendplusinf$ $0$ $\Longleftrightarrow$ \smallbreak $\forall$ $j \in \left\{ 1,\text{ }...,\text{ }p \right\} $  $(x_{j,n})_{n \in \mathbb{N}}$ vers $0$ (où $k$ $=$ $1$, $2$ ou $\infty$). Montrons $\Longrightarrow$. Supposons ||$u_n$||$_{\infty}$ $\tendplusinf$  $0$. Or, \smallbreak ||$u_n$||$_{\infty}$ $=$ Max[|$x_{1,n}$|, |$x_{2,n}$|, ..., |$x_{p,n}$|]. Alors |$x_{j,n}$| $\leqslant$ ||$u_n$||$_{\infty}$ $\tendplusinf$ $0$. Donc, $x_{j,n}$ $\tendplusinf$ $0$,\smallbreak $\forall$ $j \in \left\{ 1,\text{ }...,\text{ }p \right\} $.
\smallbreak
Montrons $\Longleftarrow$, On a : $x_{j,n}$ $\tendplusinf$ $0$,  $\forall$ $j \in \left\{ 1,\text{ }...,\text{ }p \right\} $. \smallbreak Alors ||$u_n$||$_1$ $=$ |$x_{1,n}$| $+$ |$x_{2,n}$| $+$ ... $+$ |$x_{p,n}$| $\tendplusinf$ $0$ $+$ $0$ $+$ ... $+$ $0$ $=$ $0$. Donc ||$u_n$||$_1$ $\tendplusinf$ $0$.

\subsection{Applications Continues}
\parindent=0cm
\fbox{
	\begin{minipage}{0.9\textwidth}
		\underline{Définition :} \smallbreak
		Soient ($E$, $N$), ($E'$, $N'$) deux e.v.n. Soient $A \subset E$, $f$ : $A$ $\longrightarrow$ $E'$, $a$ $\in$ $A$. On dit que $f$ est continue en $a$ ssi $\forall$ $\varepsilon >0$, $\exists$ $\eta$ $>0$ et $\forall$ $x$ $\in$ $A$, $\underset{x \in B_{E}(a,\text{ }\eta)}{\underbrace{N(x - a)< \eta}}$ $\Longrightarrow$ $\underset{f(x) \in B_{E'}(f(a),\text{ }\varepsilon)} {\underbrace{N'(f(x) - f(a)) < \varepsilon}}$.
	\end{minipage}
}


\smallbreak
\underline{Remarque :} \parindent=1cm \smallbreak
$\Longrightarrow$ $\forall$ $\varepsilon >0$, $\exists$ $\eta >0$ et $\forall$ $x$ $\in$ $A\cap B_E(a,\text{ }\eta)$, on a $f(x)$ $\in$ $B_{E'}(f(a),\text{ }\varepsilon)$
\medbreak
\begin{tikzpicture}[scale=0.75]
    \draw (0,0) ..controls +(6.5,0) and +(4,0).. (0,3);
    \draw (3.5,2) node{\footnotesize{$\bullet$}};
    \draw (3.5,2) node[below left]{\small{$\eta$}};
    \draw (3.5,2) node[above]{\small{$a$}};
    \draw (10.5,2.75) node[]{\small{$>$}};
    \draw [<->](3.5,2) -- (3.5,1.25);
    \draw (0.5,1.5) node{\Large{$A$}};
    \draw (3.5,2) circle (0.75);
    \draw (18,1.5)[pattern=north west lines] circle (0.75);
    \draw (18,1.5) circle (2);
    \draw (18,1.5) node{\footnotesize{$\bullet$}};
    \draw (18,0.75) node[below left]{\small{$\varepsilon$}};
    \draw (18,1.5) node[above]{\small{$f(a)$}};
    \draw [<->](18,1.5) -- (18,-0.5);
    \draw (3.5,2) ..controls +(1,1) and +(-1,1).. (17.5,2);
\end{tikzpicture}
\smallbreak
\parindent=0cm
\smallbreak
\underline{Remarque :} \parindent=1cm \smallbreak
Cela généralise la définition des fonctions continues d'une variable : $I$ intervalle de $\mathbb{R}$, $a$ $\in$ $I$, $f$ est continue \smallbreak en $a$ : $\forall$ $\varepsilon >0$, $\exists$ $\eta$ $>0$ et $\forall$ $x$ $\in$ $I$, $x \in$ |$x$ $-$ $a$| $<\eta$ $\Longrightarrow$ |$f(x)$ $-$ $f(a)$| $< \varepsilon$.

\parindent=0cm
\smallbreak
\underline{Proprosition :} \parindent=1cm \smallbreak
Soit $f$ : $A$ $\longrightarrow$ $E'$ continue en $a$ $\in$ $E$ . Alors, pour toute suite $(x_n)_{n \in \mathbb{N}}$ de $A$ convergeant vers $a$, la suite \smallbreak $f((x_n))_{n \in \mathbb{N}}$ converge vers $f(a)$ dans $E'$. 


\parindent=0cm
\smallbreak
\underline{Démo :} \parindent=1cm \smallbreak
Soit $\varepsilon>0$. Comme $f$ est continue en $a$, $\exists$  $\eta$ $>0$ tel que $\forall$ $x$ $\in$ $A$, $N(x$ $-$ $a)$ $<\eta$ $\Longrightarrow$ $N'(f(x)$ $-$ $f(a))$ $< \varepsilon$.
\smallbreak 
Soit ($x_n$) convergeant vers $a$, $x_n$ $\in$ $A$. Cela signifie que $N(x_n$ $-$ $a)$ $\tendplusinf$ $0$. Il existe donc $n_0$, tel que \smallbreak$\forall$ $n$ $\geqslant$ $n_0$, $N'(f(x)$ $-$ $f(a))$ $< \varepsilon$. On a prouvé que $\forall$ $\varepsilon$ $>0$, $\exists$  $n_0$, et $\forall$ $n$ $\geqslant$ $n_0$, $N'(f(x)$ $-$ $f(a))$ $< \varepsilon$.
\smallbreak Donc, $N'(f(x_n)$ $-$ $f(a))$ $\longrightarrow$ $0$ si $n\longrightarrow +\infty$ donc $f(x_n)$ $\longrightarrow$ $f(a)$ dans $E'$.

\parindent=0cm
\fbox{
	\begin{minipage}{0.9\textwidth}
		\underline{Théorème :} \smallbreak
		Soient ($E$, $N$), ($E'$, $N'$) deux e.v.n. Soient $A \subset E$, $f$ : $A$ $\longrightarrow$ $E'$, $a$ $\in$ $A$. \smallbreak $f$ est continue en $a$ $\Longleftrightarrow$ Pour toute suite $(x_n)_{n \in \mathbb{N}}$ de $A$ convergeant vers $a$, la suite $f(x_n)_{n \in \mathbb{N}}$ converge vers $f(a)$.
	\end{minipage}
}

\parindent=0cm
\smallbreak
\underline{Démo :} \parindent=1cm \smallbreak
$i$/ $\longrightarrow$ $ii$/ est la proposition précédente.
\smallbreak
On peut montrer $ii$/ $\longrightarrow$ $i$/ par contraposée. On suppose donc : $\exists$ $\varepsilon_0 >0$ et $\forall$ $\eta$ $>0$, $\exists$ $x$ $\in$ $A$ avec \smallbreak $N(x - a)< \eta$ et $N'(f(x) - f(a)) \geqslant \varepsilon_0$.
\smallbreak
Appliquons cela avec $\eta$ $=$ $\frac{1}{n+1}$ $(n \in \mathbb{N})$. Il existe donc pour tout $n \in \mathbb{N}$, $x_n$ $\in$ $A$ vérifiant $N(x_n - a)< \frac{1}{n+1}$ et \smallbreak $N'(f(x) - f(a)) \geqslant \varepsilon_0$.

\smallbreak
On a donc $N(x_n - a)$ $\tendplusinf$ $0$, donc $(x_n)_{n \in \mathbb{N}}$ est une suite de $A$ convergeant vers $a$. De plus la suite \smallbreak $f(x_n)_{n \in \mathbb{N}}$ ne peut converger vers $f(a)$ (puisque si elle convergeait vers $f(a)$, $0$ $=$ \liminfty $N'(f(x) - f(a)) \geqslant \varepsilon_0$ \smallbreak $>0$ : absurde.) On a montré que $i/$ (faux) $\Longrightarrow$ $ii/$ (faux)

\parindent=0cm
\smallbreak
\underline{Application du théorème précédent :}
\smallbreak
\underline{Proposition :} \parindent=1cm \smallbreak
Soient $E$, $E'$ deux e.v, $N$, $N_1$ deux normes équivalentes sur $E$, et $N'$, $N'_1$ deux normes équivalentes sur $E'$.
\smallbreak
Soient $A$ $\subset$ $E$, $a$ $\in$ $A$, $f$ : $A$ $\longrightarrow$ $E'$. Il y a équivalence entre :
\smallbreak $-$ $i/$ $f$ est continue en $a$ lorsque $E$ est muni de $N$, et $E'$ est muni de $N'$.
\smallbreak $-$ $ii/$ $f$ est continue en $a$ lorsque $E$ est muni de $N_1$, et $E'$ est muni de $N'_1$.
\parindent=0cm
\smallbreak
\underline{Démo :} \parindent=1cm \smallbreak
Par le théorème précédent, $i/$ équivaut à :
\smallbreak $-$ $i'/$ $\forall$ $(x_n)_{n \in \mathbb{N}}$ de $A$ vérifiant $N(x_n - a)$ $\longrightarrow $ $0$, on a : $N'(f(x) - f(a))$ $\longrightarrow $ $0$. \smallbreak De même $ii/$ équivaut à :
\smallbreak $-$ $ii'/$ $\forall$ $(x_n)_{n \in \mathbb{N}}$ de $A$ vérifiant $N_1(x_n - a)$ $\longrightarrow $ $0$, on a : $N_1'(f(x) - f(a))$ $\longrightarrow $ $0$.
\smallbreak Or, on a vu que si $N$ est équivalente à $N_1$: ($N(x_n - a)$ $\longrightarrow $ $0$) $\Longleftrightarrow$ ($N_1(x_n - a)$ $\longrightarrow $ $0$). De même, comme \smallbreak $N'$ est équivalente à $N'_1$, ($N'(f(x) - f(a))$ $\longrightarrow $ $0$) $\Longleftrightarrow$ ($N_1'(f(x) - f(a))$ $\longrightarrow $ $0$). Donc $i'/$ $\Longleftrightarrow$ $ii'/$.

\parindent=0cm
\smallbreak
\underline{Remarque :} \parindent=1cm \smallbreak
Supposons $E$ $=$ $\mathbb{R}^n$, $E'$ $=$ $\mathbb{R}^p$. On sait que les normes ||..||$_1$, ||..||$_2$, ||..||$_{\infty}$, sont équivalentes. Lorsqu'on \smallbreak étudie la continuité de $f$ : $A$ $\longrightarrow$ $E'$. où $A$ $\subset$ $E$, on peut étudier n'importe laquelle de ces normes.

\parindent=0cm
\subsection{Sommes de fonctions continues en un point}
\underline{Notation :} \parindent=1cm \smallbreak $f$ : $A$ $\longrightarrow$ $E'$, $g$ : $A$ $\longrightarrow$ $E'$, on note $f$ $+$ $g$ : $A$ $\longrightarrow$ $E'$, $x$ $\mapsto$ $(f$ $+$ $g)(x)$ $=$ $f(x)$ $+$ $g(x)$. Si $\lambda$ $\in$ $\mathbb{R}$, on pose \smallbreak $(\lambda \cdot f)(x)$ $:=$ $\lambda f(x)$, $\forall$ $x$ $\in$ $a$.
\parindent=0cm
\smallbreak
\underline{Proposition :} \parindent=1cm \smallbreak
Si $f$ et $g$ sont continues en $a$ $\in$ $A$ alors $f$ $+$ $g$ et $\lambda f(x)$ sont continues en $a$.

\parindent=0cm
\smallbreak
\underline{Démo :} \parindent=1cm \smallbreak Pour voir que $f$ $+$ $g$ est continue en $a$, il suffit de montrer que $\forall$ $(x_n)_{n \in \mathbb{N}}$ de $A$ convergeant vers $a$, \smallbreak $((f$ $+$ $g)(x_n))_{n \in \mathbb{N}}$ converge vers $(f$ $+$ $g)(a)$. Or comme $f$ est continue en $a$, $f(x_n)$ $\longrightarrow$ $f(a)$ \smallbreak et $g$ est continue en $a$, $g(x_n)$ $\longrightarrow$ $g(a)$. Donc $f(x_n)$ $+$ $g(x_n)$ $\longrightarrow$ $f(a)$ $+$ $g(a)$.

\parindent=0cm
\smallbreak
\underline{Proposition :} \parindent=1cm \smallbreak
Soient ($E$, $N$), ($E'$, $N'$), ($E''$, $N''$) trois e.v.n. Soient $A$ $\subset$ $E$, $B$ $\subset$ $E'$, 
$f$ : $A$ $\longrightarrow$ $E'$, $g$ : $B$ $\longrightarrow$ $E''$. \smallbreak Supposons $f(A)$ $\subset$ $B$. On peut donc définir $g$ $\circ$ $f$ : $A$ $\longrightarrow$ $E''$. Soit $a$ $\in$ $A$. On pose $b$ $=$ $f(a)$ ($\in$ $f(A)$ $\subset$ $B$) \smallbreak
Supposons $f$ continue en $a$ et $g$ continue en $b$. Alors $g$ $\circ$ $f$ continue en $g$ $\circ$ $f$ est continue en $a$.

\parindent=0cm
\smallbreak
\underline{Démo :} \parindent=1cm \smallbreak
Il suffit de voir que pour toute quite $(x_n)_{n \in \mathbb{N}}$ de $A$ convergeant vers $a$ $((g$ $\circ$ $f)(x_n))_{n \in \mathbb{N}}$  converge vers \smallbreak $(g$ $\circ$ $f)(a)$. Or comme $f$ est continue en $a$, $x_n$ $\longrightarrow$ $a$ $\Longleftrightarrow$ $a$, $f(x_n)$ $\longrightarrow$ $b$, et comme $g$ est continue en $b$, \smallbreak $y_n = f(x_n)$ $\longrightarrow$ $b$ $\Longleftrightarrow$ $g(y_n) =$ $(g$ $\circ$ $f)(x_n)$ $\longrightarrow$ $g(b)$ $=$ $(g$ $\circ$ $f)(a)$

\parindent=0cm
\smallbreak
\underline{Proposition :} \parindent=1cm \smallbreak
Soient $(E$, $N)$ un e.v.n. $A$ $\subset$ $E$, $a$ $\in$ $E$, $f$ : $A$ $\longrightarrow$ $\mathbb{R}^p$. On munit $\mathbb{R}^p$ de l'une des normes ||..||$_1$, ||..||$_2$, ||..||$_{\infty}$. \smallbreak Pour $x$ $\in$ $A$, écrivons $f(x)$ $=$ $
\begin{bmatrix}
	f_1(x) \\
	\vdots \\
	f_p(x)
\end{bmatrix}
$ $\in$ $\mathbb{R}^p$ On obtient $f_j$ : $A$ $\longrightarrow$ $\mathbb{R}$, $j \in \left\{ 1,\text{ }...,\text{ }p \right\} $. Il y a équivalence entre :
\smallbreak 
$-$ $i/$ $f$ est continue en $a$. \smallbreak
$-$ $ii/$ $\forall$ $j \in \left\{ 1,\text{ }...,\text{ }p \right\} $, $f_j$ est continue en $a$.

\parindent=0cm
\smallbreak
\underline{Démo :} \parindent=1cm \smallbreak
$-$ $i/$ $\Longleftrightarrow$ 
$(x_n)_{n \in \mathbb{N}}$ de $A$ $\longrightarrow$ $a$, la suite $f((x_n))_{n \in \mathbb{N}}$ converge vers $f(a)$ dans $\mathbb{R}$.
\smallbreak
$-$ $ii/$ $\Longleftrightarrow$ 
$(x_n)_{n \in \mathbb{N}}$ de $A$ $\longrightarrow$ $a$, $\forall$ $j \in \left\{ 1,\text{ }...,\text{ }p \right\} $ la suite $f_j((x_n))_{n \in \mathbb{N}}$ $\longrightarrow$ $f_j(a)$.
\smallbreak
Or on a vu que $
\begin{bmatrix}
	f_1(x_n) \\
	\vdots \\
	f_p(x_n)
\end{bmatrix}
$ $\longrightarrow$ 
$
\begin{bmatrix}
	f_1(a) \\
	\vdots \\
	f_p(a)
\end{bmatrix}
$
$\Longleftrightarrow$ $\forall$ $j \in \left\{ 1,\text{ }...,\text{ }p \right\}$ $f_j((x_n))_{n \in \mathbb{N}}$ $\tendplusinf$ $f_j(a)$.

\parindent=0cm
\fbox{
	\begin{minipage}{0.9\textwidth}
		\underline{Définition :} \smallbreak
		On dit que $f$ : $A$ $\longrightarrow$ $E'$ est continue sur $A$ $\Longleftrightarrow$ $\forall$ $a$, $\in$ $A$ $f$ est continue est $a$.
	\end{minipage}
}
\smallbreak
\parindent=0cm
\fbox{
	\begin{minipage}{0.9\textwidth}
		\underline{Théorème :} \smallbreak
		Soient ($E$, $N$), ($E'$, $N'$) deux e.v.n. Soient $u$ : $E$ $\longrightarrow$ $E'$ linéaire. Il y a équivalence entre : \smallbreak
		$-$ $i/$ $u$ est continue sur $E$ \smallbreak
		$-$ $ii/$ $u$ est continue sur $0$ \smallbreak
		$-$ $iii/$ $\exists$ $C>0$, $\forall$ $x$ $\in$ $E$, $N'(u(x))$ $\leqslant$ $CN(u(x))$.
	\end{minipage}
}
\parindent=0cm
\smallbreak
\underline{Démo :} \parindent=1cm \smallbreak
$i/$ $\Longrightarrow$ $ii/$ : évident. $0$ $\in$ $E$ \smallbreak
$ii/$ $\Longrightarrow$ $iii/$ : Si $u$ est continue en $0$, $\forall$ $\varepsilon >0$ et $\exists$ $\eta$ $>0$ et $\forall$ $x$ $\in$ $E$, $N(x$ $-$ $0)$ $<$ $\eta$ $\Longrightarrow$ $N'(u(x)$ $-$ $u(0))$ $<$ $\varepsilon$
\smallbreak
Comme $u$ est linéaire, $u(0)$ $=$ $0$ donc $N(x)$ $<$ $\eta$ $\Longrightarrow$ $N'(u(x))$ $<$ $\varepsilon$
Appliquons cela avec $\varepsilon$ $=$ $1$. \smallbreak $\exists$ $\eta_0$ $>0$ tel que $N(x)$ $<\eta_0$ $\Longrightarrow$ $N'(u(x))$ $<$ $1$. Soit $y$ $\in$ $E$, $y$
$\neq$ $0$. Posons $x$ $=$ $\frac{y}{N(y)}\cdot \frac{\eta_{\footnotesize{0}}}{2}$. Alors $N(x)$ $=$  $\frac{\eta_0}{2}$ \smallbreak $<$ $\eta_0$ donc soit $N'(\frac{\eta_0}{2N(y)}u(y))$ $<1$ puisque $u$ est linéaire. Donc, $\forall$ $y$ $\in$ $E \smallsetminus \left\{ 0 \right\}$ $\frac{N'(u(y))}{2N(y)/ \eta_0}$ $<1$ d'où \smallbreak $N'(u(y))$ $<$ $CN(y)$ avec $C = \frac{2}{\eta_0}$
Donc $iii/$ est vraie.
\smallbreak
$iii/$ $\Longrightarrow$ $i/$ : Soit $a$ $\in$ $E$. On veut monter que $u$ est continue en $a$. $\exists$ $C>0$ et $\forall$ $y$ $\in$ $E$, $N'(y)$ $<$ $CN(y)$.
\smallbreak
Pour $\varepsilon$ $>0$ donné, posons $\eta$ $=$ $\varepsilon/C$, supposons $N(x$ $-$ $a)$ $<\eta$. Alors, \smallbreak $N'(u(x)$ $-$ $u(a))$ $=$ $N'(u(x$ $-$ $a))$ $\leqslant$ $CN(x$ $-$ $a)$ $<$ $C\eta$ $=$ $\varepsilon$. Donc $u$ est continue en $a$.

\parindent=0cm
\smallbreak
\underline{Proposition :} \parindent=1cm \smallbreak
Munissons $\mathbb{R}^p$ de l'une des normes ||..||$_1$, ||..||$_2$, ||..||$_{\infty}$. Soient $(E$, $N)$ un e.v.n et $u$ : $\mathbb{R}^p$ $\longrightarrow$ $E$, une \smallbreak application linéaire. Alors $u$ est continue.
\parindent=0cm
\smallbreak
\underline{Démo :} \parindent=1cm \smallbreak
Notons ($e_1$, ..., $e_p$) la base canonique de $\mathbb{R}^p$, si $x$ $\in$ $\mathbb{R}^p$, $\overset{p}{\underset{j = 1}{\sum}}x_j e_j$. Alors $u(x)$ $=$ $\overset{p}{\underset{j = 1}{\sum}}x_j u(e_j)$ car $u$ est linéaire. \smallbreak Alors $N(u(x))$ $=$ $N( \overset{p}{\underset{j = 1}{\sum}}x_j u(e_j))$ $\leqslant$ 
$ \overset{p}{\underset{j = 1}{\sum}}N(x_j u(e_j))$ $=$ $ \overset{p}{\underset{j = 1}{\sum}}|x_j|N(u(e_j))$. Posons $M$ $=$ Max[$N'(u(e_j))$]. Donc \smallbreak $N(u(x))$ $\leqslant$ $M \overset{p}{\underset{j = 1}{\sum}}|x_j|$ $=$ $M||x||_1$. D'après $iii/$ de la propriété pécédente, cela implique que $u$ est continue.

\subsection{Exemples de fonctions continues}
\parindent=0cm
\smallbreak
\underline{Ex 1 :} \parindent=1cm \smallbreak
L'application $E$ $\times$ $E$ $\longrightarrow$ $E$, $(x$, $y)$ $\mapsto$ $x$ $+$ $y$ est continue.

\parindent=0cm
\smallbreak
\underline{Ex 2 :} \parindent=1cm \smallbreak
L'application $\mathbb{R}$ $\times$ $\mathbb{R}$ $\longrightarrow$ $\mathbb{R}$, ($x$, $y$) $\mapsto$ $x$ $\cdot$ $y$ est continue.

\parindent=0cm
\smallbreak
\underline{Ex 3 :} \parindent=1cm \smallbreak
L'application $\mathbb{C}$ $\times$ $\mathbb{C}$ $\longrightarrow$ $\mathbb{C}$, ($z$, $\omega$) $\mapsto$ $z$ $\cdot$ $\omega$ est continue.

\parindent=0cm
\smallbreak
\underline{Ex 4 :} \parindent=1cm \smallbreak
Soient $(E$, $N)$ un e.v.n, $A$ $\subset$ $E$, $f$ : $A$ $\longrightarrow$ $\mathbb{R}$*. Soit $a$ $\in$ $E$, supposons $f$ continue en $a$. Alors, $x$ $\mapsto$ $\frac{1}{f(x)}$ \smallbreak est continue en $a$.

\parindent=0cm
\smallbreak
\underline{Ex 5 :} \parindent=1cm \smallbreak
Soit $E$ $=$ $M_p(\mathbb{R})$ l'ensemble des matrices carrées d'ordre $p$. Si $A$ $=$ $(a_{ij})_{1 \leqslant i,j \leqslant p}$ $\in$ $M_p(\mathbb{R})$, posons \smallbreak $N(A)$ $=$ $p\underset{1 \leqslant i,j \leqslant p}{\text{Max}}$ |$a_{ij}$|. $N$ est une norme et on pose $N'$ une norme sur $E$, telle que $\forall$ $A$, $B$ $\in$ $M_p(\mathbb{R}$ \smallbreak $N'((A$, $B))$ $=$ Max[$N(A)$, $N(B)$]. Soit $\Phi$ : $E$ $\times$ $E$ $\longrightarrow$ $E$. ($A$, $B$) $\longrightarrow$ $AB$. $\Phi$ est une continue dans $(E$, $N')$.

\parindent=0cm
\smallbreak
\underline{Ex 6 :} \parindent=1cm \smallbreak
Soit $(E$, $N)$ un e.v.n. L'application $E$ $\longrightarrow$ $\mathbb{R}$, $x$ $\mapsto$ $N(x)$ est continue.
\smallbreak
(Je scannerai leurs démonstrations car trop longues pour certaines.)
\smallbreak

\subsection{Exemples de fonctions non continues}

Soit $f$ : $\mathbb{R}^2$ $\longrightarrow$ $\mathbb{R}^2$, $(x$, $y)$ $\mapsto$ $\frac{xy}{x^2+y^2}$ si $(x$, $y)$ $\neq$ $0$, et $f(0$, $0)$ $=$ $0$. 
\parindent=0cm
\smallbreak
\underline{Démo :} \parindent=1cm \smallbreak
Montrons que $f$ n'est pas continue en $(0$, $0)$ \smallbreak. Il suffit de construire une suite $((x_n$, $y_n))_{n \in \mathbb{N}}$ convergeant vers $0$, telle que $f(x_n$, $y_n)$ ne converge pas \smallbreak vers $f(0$, $0)$ $=$ $0$. Prenons, $x_n$ $=$
$y_n$ $=$ $\frac{1}{n+1}$ Donc  $(x_n$, $y_n)$ $\mapsto$ $(0$, $0)$. Mais \smallbreak
$f(x_n$, $y_n)$ $=$ $\frac{\frac{1}{1+n} \cdot \frac{1}{1+n}}{\left( \frac{1}{1+n} \right)^2 + \text{ } \left( \frac{1}{1+n} \right)^2}$ $=$ $\frac{1}{2}$. En particulier $f(x_n$, $y_n)$ ${\Huge \nrightarrow}$ $0$. 


Si on fixe $y$ $=$ $y_0$, $x$ $\mapsto$ $f(x$, $y_0)$ est continue \smallbreak 
$-$ Si $y_0$ $\neq$ $0$ : $f(x$, $y_0)$ $=$ $\frac{xy_0}{x^2+y_0^2}$ et le dénominateur ne s'annule pas.

$-$ Si $y_0$ $=$ $0$ : $f(x$, $0)$ $=$ $0$. $\forall$ $x$ $\in$ $\mathbb{R}$ donc $x$ $\mapsto$ $f(x$, $0)$ est continue.
\smallbreak
Donc $f$ est continue séparément par rapport à chaque variable, mais pas comme fonction de deux variables.

\section{Topologie sur un e.v.n}
\subsection{Ouverts d'un e.v.n}

\parindent=0cm
\fbox{
	\begin{minipage}{0.9\textwidth}
		\underline{Définition :} \smallbreak
		Soit $(E$, $N)$, un e.v.n. On dit que $U$ $\subset$ $E$ est ouvert $\Longleftrightarrow$ $\forall$ $a$ $\in$ $U$, $\exists$ $r>0$, tel que $B(a$, $r)$ $\subset$ $U$.
	\end{minipage}
}
\parindent=0cm
\smallbreak
\underline{Ex :} \parindent=1cm \smallbreak

\begin{minipage}{0.2\linewidth}
    \begin{tikzpicture}[scale=0.75]
        \draw (1,2) ..controls +(1.5,1.5) and +(0.5,2).. (4,1);
        \draw (1,2) ..controls +(-1,-1) and +(-0.1,0).. (1,1);
        \draw (1,1) ..controls +(0.5,0) and +(-0.5,0.5).. (3,0.5);
        \draw (3,0.5) ..controls +(0.5,-0.5) and +(-0.25,-1).. (4,1);
        \draw (3,2) node{\footnotesize{$\times$}};
        \draw (3,2) node[below left]{\small{$r$}};
        \draw (2,1.5) node{\footnotesize{$B(a,r)$}};
        \draw (3,2) node[above]{\small{$a$}};
        \draw [<->](3,2) -- (3,1.5);
        \draw (0.5,0.5) node{\Large{$U$}};
        \draw (3,2) circle (0.5);
    \end{tikzpicture}
\end{minipage}\begin{minipage}{0.8\linewidth}
    $-$ $U$ est ouvert. \smallbreak
    $-$ $U$ $=$ $\emptyset$ est ouvert.


\end{minipage}
\parindent=0cm
\smallbreak
\underline{Ex :} \parindent=1cm \smallbreak
$-$ $E$ $=$ $\mathbb{R}$, $N(x)$ $=$ |$x$|. Soit $I$ $=$
]$\alpha$, $\beta$[, $\alpha$, $\beta$ $\in$ $\mathbb{R}$, (ou $\alpha$ $=$ $-\infty$, ou $\beta$ $=$ +$\infty$). Alors $I$ est ouvert. Soit $a$ $\in$ $I$.
\smallbreak
\begin{picture}(5.5,0.55)(0,0.15) \put(0,0.25){\line(1,0){5.5}}
\put(0.4,0.16){]}
\put(1.5,0.16){]}
\put(4.8,0.16){[}
\put(3.7,0.16){[}
\put(2.6,0.16){|}
\put(1.1,0.5){\footnotesize{$a$ $-$ $r$}}
\put(0.4,0.5){\footnotesize{$\alpha$}}
\put(4.75,0.5){\footnotesize{$\beta$}} 
\put(2.55,0.5){\footnotesize{$a$}}
\put(3.3,0.5){\footnotesize{$a$ $+$ $r$}}

\end{picture}
On doit trouver $r>0$, tel que $B(a$, $r)$ $=$ ]$a-r$, $b-r$[ $\notin$ $I$. Il suffit de prendre \smallbreak $r$ $<$ Min [$a-\alpha$, $b-\beta$].
\smallbreak
$-$ [$0, 1$[ n'est pas un ouvert.  On ne peut pas trouver un $r>0$, tel que : $B(0$, $r)$ $\subset$ [$0, 1$[.
\parindent=0cm
\smallbreak
\underline{Notation :} \parindent=1cm \smallbreak
Soit ($A_i$)$_{i\in I}$ une famille de s.e.v. de $E$.
On note $\underset{i\in I}{\bigcup}$ $A_i$ $=$ $\left\{ x \in E, \exists \text{ } i \in I, \text{tel que } x \in A_i \right\}.$

\parindent=0cm
\smallbreak
\underline{Proposition :} Soit $(E$, $N)$ un e.v.n. \parindent=1cm \smallbreak
$i/$ Soit ($u_i$)$_{i\in I}$, une famille d'ouverts de $E$. Alors $\underset{i\in I}{\bigcup}$ $u_i$ est un ouvert de $E$. \smallbreak
$ii/$ Soient $U$,$V$ deux ouverts de $E$. Alors $U \cap V$ est ouvert. Plus généralement, si $U_1$, $U_2$, ..., $U_n$ sont des \smallbreak ouverts de $E$ alors $U_1$ $\cap$ $U_2$ $\cap$ ... $\cap$ $U_n$ est un ouvert de $E$. (Toute intersection finie d'éléments est un ouvert.)

\parindent=0cm
\smallbreak
\underline{Démo :} \parindent=1cm \smallbreak
$i/$ Supposons $\forall$ $i$ $\in$ $I$. $U_i$ est ouvert. Soit $U$ $=$ $\underset{i\in I}{\bigcup}$ $U_i$. Soit $a$ $\in$ $U$. $\exists$ $i_0$ $\in$ $I$, tel que $a$ $\in$ $U_{i_{0}}$. Comme $U_{i_{0}}$ est \smallbreak ouvert $\exists$ $r>0$, tel que $B(a$, $r)$ $\subset$ $U_{i_{0}}$.
$U_{i_{0}}$ $\subset$ $\underset{i\in I}{\bigcup}$ $U_i$ $=$
$U$ donc on a trouvé $r>0$ avec $B(a$, $r)$ $\subset$ $U$.
Donc $U$ \smallbreak est ouvert.
\smallbreak
$ii/$ Soient $U$, $V$ deux ouverts de $E$. Montrons que $U$ $\cap$ $V$ est ouvert. Soit $a$ $\in$ $U$ $\cap$ $V$. Comme $a$ $\in$ $U$ et que \smallbreak $U$ est ouvert, $\exists$ $r_1>0$, avec $B(a$, $r_1)$ $\subset$ $U$. De même, comme $a$ $\in$ $V$. Posons $r$ $=$ Min[$r_1$, $r_2$] > 0.
\smallbreak Alors $B(a$, $r)$ $\subset$ $B(a$, $r_1)$ $\subset$ $U$ et  
$B(a$, $r)$ $\subset$ $B(a$, $r_2)$ $\subset$ $V$. $\Longrightarrow$ $B(a$, $r)$  $\subset$ $U$ $\cap$ $V$.
\parindent=0cm
\smallbreak
\underline{Attention :} \parindent=1cm \smallbreak
L'intersection d'une famille infinie n'est pas ouvert en général.
\parindent=0cm
\smallbreak
\underline{Ex :} \parindent=1cm \smallbreak
Soient $U_n$ $=$ $\left] -\frac{1}{n}, 1 \right[$ $n \in \mathbb{N}^*$ et $A$ $=$ $\overset{+\infty}{\underset{n \in \mathbb{N}}{\bigcap}}U_n$ $=$  $\left\{ x \in E;\text{ } \forall n \in \mathbb{N}^*, \text{ } x \in U_n \right\}$ $=$ $\left\{ x \in E;\text{ } \forall n \in \mathbb{N}^*, -\frac{1}{n} < x < 1\right\}$. \smallbreak Donc $A$ $=$ [0, 1[. Or, on a vu que $A$ n'est pas ouvert.

\parindent=0cm
\smallbreak
\underline{Notation :} \parindent=1cm \smallbreak
Si $X$, $Y$ sont des ensembles, et $f$ : $X$ $\longrightarrow$  $Y$, si $Z$ $\subset$ $Y$, on pose $f^{-1}(Z)$ $\overset{ def }{=}$ $\left\{ x \in E;\text{ } f(x) \in Z \right\}$. On note \smallbreak aussi $f^{-1}(Z)$ $=$ $f^{*}(Z)$.

\parindent=0cm
\smallbreak
\underline{Proposition :} \parindent=1cm \smallbreak
Soient $(E$, $N)$, $(E'$, $N')$ deux e.v.n. $f$ : $E$ $\rightarrow$ $E'$ continue. Si $V$ est ouvert de $E$, $f^{-1}(V)$ est un ouvert de $E$.

\parindent=0cm
\smallbreak
\underline{Démo :} \parindent=1cm \smallbreak
\begin{minipage}{0.6\linewidth}
    Posons $U$ $=$ $f^{-1}(V)$. Soit $a$ $\in$ $U$. On doit montrer qu'il existe $r>0$, tel que $B(a$, $r)$ $\subset$ $U$. Notons $b$ $=$ $f(a)$ $\in$ $V$. Comme $V$ est ouvert \smallbreak $\exists$ $\varepsilon>0$, tel que $B(b$, $\varepsilon)$ $\subset$ $V$. Comme $f$ est continue en $a$, $\exists$ $\eta>0$ tel que $\forall$ $x$ $\in$ $E$, $N(x$ $-$ $a)$ $\Longrightarrow$ $N'(f(x)$ $-$ $b)$ $<\varepsilon$. $x$ $\in$ $B_E(a$, $\eta)$ $\Longrightarrow$ $f(x)$ $\in$ $B_{E'}(b$, $\varepsilon)$ $\subset$ $V$. Donc $\forall$ $x$  $\in$ $B(a$, $\eta)$, $f(x)$ $\in$ $V$, 
\end{minipage}\begin{minipage}{0.4\linewidth}
    \begin{tikzpicture}[scale=1]
        \draw (0,0) node{\tiny{ }};
        \draw (1.25,0) circle (1);
        \draw (1.6180,0.4) node{$\times$};
        \draw (4.6180,-0.2) node{$\times$};
        \draw (1.6180,0.4) node[above]{$a$};
        \draw (4.6180,-0.2) node[above]{$b$};
        \draw (3.9,-0.25) node{\large{$V$}};
        \draw (0.7,-0.25) node{\large{$U$}};
        \draw (4.5,0) circle (1);
    \end{tikzpicture}
\end{minipage}
\hspace*{1cm}donc $B(a$, $\eta)$ $\subset$ $f^{-1}(V)$. Si on pose $r$ $=$ $\eta$, on aura donc trouvé une boule $B(a$, $r)$ $\subset$ $f^{-1}(V)$ $=$ $U$. Donc \smallbreak $f^{-1}(V)$ est ouvert.

\parindent=0cm
\smallbreak
\underline{Application :} \parindent=1cm \smallbreak
Soit $a$ $\in$ $E$, $r>0$. Alors la boule ouverte $B(a$, $r)$ est un ouvert $E$. Soit $f$: $E$ $\longrightarrow$ $\mathbb{R}$, $x$ $\mapsto$ $N(x$, $a)$. On a vu \smallbreak que $f$ est continue. Or $B(a$, $r)$ $=$ $\left\{ x \in E,\text{ } f(x) < r \right\}$ $=$ $f^{-1}\underset{\text{ouvert de} \mathbb{R}}{\underbrace{(]-\infty; r[)}}$ $\Longrightarrow$ $f^{-1}(]-\infty; r[)$ est un ouvert.
\smallbreak
De même $E \smallsetminus$\Bbarre$(a$, $r)$ est un ouvert car $E \smallsetminus$\Bbarre$(a$, $r)$ = $\left\{ x \text{ }\in \text{ }E,\text{ }N(x\text{ }-\text{ }a)>r \right\}$ $=$ $\left\{ x \text{ }\in \text{ }E,f(x)>\text{ }r \right\}$ $=$ \smallbreak  $f^{-1}(]-\infty; r[)$.

\parindent=0cm
\smallbreak
\underline{Proposition :} \parindent=1cm \smallbreak
Soient $N_1$, $N_2$ deux normes équivalentes sur un e.v. $E$. Soit $U$ $\subset$ $E$, $U$ est ouvert pour $N_1$ $\Longleftrightarrow$ $U$ est ouvert \smallbreak pour $N_2$.

\parindent=0cm
\smallbreak
\underline{Démo :} \parindent=1cm \smallbreak
$N_1$, $N_2$ équivalents, $\exists$ $C>0$ avec $\forall$ $x$ $\in$ $E$, $N_1(x)$ $\leqslant$ $CN_2(x)$ et $N_2(x)$ $\leqslant$ $CN_1(x)$.
\smallbreak
Soit $U$ un ouvert pour $N_1$. \smallbreak
Montrons que $U$ un ouvert pour $N_2$. $\Longleftrightarrow$ $\forall$ $a$ $\in$ $U$, $\exists$ $r_2>0$, $B_{N_{2}}(a$, $r_2)$ $=$ $\underset{\subset U}{\underbrace{N(x - a)< \eta}}$
\smallbreak
Comment $U$ est ouvert pour $N_1$, on sait qu'il existe $r_1>0$, $B_{N_{1}}(a$ $-$
$r_1)$ $=$ $\left\{ x \in E,\text{ } N_1(x-a)<r_1 \right\}$ $\subset$ $U$. 
\smallbreak
Posons $r_2$ $=$ $\frac{r_1}{C}$. Soit $x$ $\in$ $B_{N_{2}}(a$, $r_2)$ $\subset$ $U$ alors $N_2(x$ $-$ $a)<r_2$ d'où $=$ $N_1(x$ $-$ $a)$ $\leqslant$ $N_2(x$ $-$ $a) \cdot C$ $<C \cdot r_2$ \smallbreak $=$ $r_1$. Donc $B_{N_{2}}(a$, $r_2)$ $\subset$ $B_{N_{1}}(a$, $r_1)$. Comme $B_{N_{1}}(a$, $r_1)$ $\subset$ $U$ on a donc trouvé $r_2>0$ avec $B_{N_{2}}(a$, $r_2)$ $\subset$ $U$.

\parindent=0cm
\smallbreak
\underline{Corollaire :} \parindent=1cm \smallbreak
Soit $E$ $=$ $\mathbb{E}^p$. Alors les ouverts associés aux normes ||..||$_1$, ||..||$_2$, ||..||$_{\infty}$ sont les mêmes.

\parindent=0cm
\smallbreak
\underline{Ex :} \parindent=1cm \smallbreak $P$ $=$ ]$a$, $b$[ $\times$ ]$c$, $d$[ $\subset$ $\mathbb{R}^2$ (muni de l'une des normes ||..||$_1$, ||..||$_2$, ||..||$_{\infty}$) alors $P$ est ouvert. En effet, \smallbreak un intervalle ouvert de $\mathbb{R}$, ]$\alpha$, $\beta$[ est un ouvert. Soient
\smallbreak$\pi_1$ : $\mathbb{R}^2$ $\xrightarrow{\hspace{0.85cm}}$ $\mathbb{R}^2$, $\pi_2$ : $\mathbb{R}$ $\xrightarrow{\hspace{0.85cm}}$ $\mathbb{R}^2$,
\smallbreak
\hspace{0.65cm}($x_1$, $x_2$) $\mapsto$ $x_1$,\hspace{0.65cm}($x_1$, $x_2$) $\mapsto$ $x_2$. 
\smallbreak
Alors $\pi_1$, $\pi_2$ sont linéaires sur $\mathbb{R}^2$. On sait alors qu'elles sont continues.
\smallbreak
Donc comme ]$a$, $b$[ $\subset$ $\mathbb{R}$ est ouvert, $\pi^{-1}_1($]$a$, $b$[$)$ $=$ $U_1$ est ouvert.
]$c$, $d$[ $\subset$ $\mathbb{R}$ est ouvert, $\pi^{-1}_2($]$c$, $d$[$)$ $=$ $U_2$ \smallbreak est ouvert.
$P$ $=$ $U_1$ $\cap$ $U_2$ est un ouvert.

\subsection{Fermés d'un e.v.n.}
\parindent=0cm
\fbox{
	\begin{minipage}{0.9\textwidth}
		\underline{Définition :} \smallbreak
		Soit $(E$, $N)$, un e.v.n. On dit que $F$ $\subset$ $E$ est fermé $\Longleftrightarrow$ $E$ $-$ $F$ est ouvert.
	\end{minipage}
}


\smallbreak
\underline{Ex :} \parindent=1cm \smallbreak
$-$ $\emptyset$ est fermé car $E - \emptyset$ $=$ $E$ qui est ouvert \smallbreak
$-$ $E$ est fermé car $E - E$ $=$ $\emptyset$ qui est ouvert.
\smallbreak
Donc $\emptyset$, et $E$ sont à la fois ouverts et fermés.
\smallbreak

$-$ \Bbarre$(a$, $r)$ $=$ $\left\{ x \in E; N(x - a) \leqslant r \right\}$ est fermé car $E -$\Bbarre$(a$, $r)$ $=$ $\left\{ x \in E; N(x - a) > r \right\}$ est ouvert.
\smallbreak 

$-$ $\left\{ x \in E; N(x - a) < r \right\}$ est fermé car $E - F$ $=$ $\left\{ x \in E; N(x - a) < r \right\}$ $=$ $B(a$, $r)$ est ouvert.
\smallbreak 
\newcommand{\R}{$\mathbb{R}$}
\newcommand{\C}{$\mathbb{C}$}
\newcommand{\NN}{$\mathbb{N}$}
\newcommand{\Nn}{$\mathbb{N}\text{ }$}
\newcommand{\Z}{$\mathbb{Z}$}
\newcommand{\Q}{$\mathbb{Q}$}
\newcommand{\D}{$\mathbb{D}$}
\newcommand{\E}{$E$ }
\newcommand{\Ep}{$E'$ }
\newcommand{\Nor}[1]{$N(#1)$}
\newcommand{\Noru}[1]{$N_1(#1)$}
\newcommand{\Nord}[1]{$N_2(#1)$}
\newcommand{\Nort}[1]{$N_3(#1)$}
\newcommand{\Norp}[1]{$N'(#1)$}
\newcommand{\Bo}[2]{$B\left( #1, #2 \right)$}
\newcommand{\Bf}[2]{\Bbarre$\left(#1, #2\right)$}
\newcommand{\N}{$N$}
\newcommand{\Np}{$N'$}
\newcommand{\Nu}{$N_1$}
\newcommand{\Nd}{$N_2$}
\newcommand{\Nt}{$N_3$}
\newcommand{\evn}{$(E$, $N)$ }
\newcommand{\evnpy}{$(E'$, $N)$ }
\newcommand{\evnyp}{$(E$, $N')$ }
\newcommand{\evnp}{$(E'$, $N')$ }
\newcommand{\evnps}{$(E'$, $N'')$ }
\newcommand{\evnsp}{$(E''$, $N')$ }
\newcommand{\evns}{$(E''$, $N'')$ }
\newcommand{\evnu}{$(E_1$, $N_1)$ }
\newcommand{\evnd}{$(E_2$, $N_2)$ }
\newcommand{\evnt}{$(E_3$, $N_3)$ }
\newcommand{\Rd}{\mathbb{R}^2}
\newcommand{\Rt}{\mathbb{R}^3}
\newcommand{\Rp}{\mathbb{R}^p}
\newcommand{\Cdeux}{\mathbb{C^2}}
\newcommand{\Ctrois}{\mathbb{C^3}}
\newcommand{\kinn}{k \in \mathbb{N}}
\newcommand{\ninn}{n \in \mathbb{N}}
\newcommand{\iini}{i \in I}
Cas particulier : \parindent=1.5cm \smallbreak $E$ $=$ \R, \Nor{x} $=$ |$x$|. Alors si $a$ < $b$ [$a$, $b$] $=$
\Bbarre($c$, $r$) où $c$ $=$ $\frac{a+b}{2}$, $r$ $=$ $\frac{b-a}{2}$. Donc [$a$, $b$] est un fermé. \smallbreak \parindent=0cm
Un s.e.v d'un e.v.n peut être ni ouvert, ni fermé.

\smallbreak
\underline{Ex :} \parindent=1cm \smallbreak 
\E$=$ \R,  $A$ $=$ [0, 1[ alors A n'est pas ouvert car on ne peut pas trouver $r>0$, ]$-r$, $r$[ $\subset$ $A$. Mais $A$ n'est pas \smallbreak non plus fermé. S'il l'était, $E - A$ serait ouvert. $E - A$ $=$ ]$-\infty$, $0$[ $\cup$ [$1$, $+\infty$[ qui n'est pas ouvert car il \smallbreak n'existe pas $r>0$ tel que \Bf{1}{r} $=$ ]$1$ $-$ $r$, $1$ $+$ $r$[ $\subset$ $E - A$.

\parindent=0cm \smallbreak
\underline{Proposition :} Soit \evn un e.v.n :\parindent=1cm \smallbreak 
$-$ $F_1$, $F_2$ deux fermés de $E$. Alors $F_1$ $\cup$ $F_2$ est fermé. Plus généralement, si $F_1$, ..., $F_n$ sont des fermés, \smallbreak alors $F_1$ $\cup$ ... $\cup$ $F_n$ est fermé.
\smallbreak 
$-$ Soit $(F_i)_{\iini}$ une famille de fermés de $E$, alors
$\underset{\iini}{\bigcap}$ $F_i$ $=$ $\left\{ x \in E, \forall \iini, x \in F_i \right\}$ est un fermé.

\parindent=0cm \smallbreak
\underline{Démo :}\parindent=1cm \smallbreak
\begin{minipage}{0.6\linewidth}
    Soient $F_1$, $F_2$ deux fermés de $E$. $U_1$ $=$ $E - F_1$ et $U_2$ $=$ $E - F_2$ ouverts de $E$.
    \smallbreak
    Alors $U_1$ $\cap$ $U_2$ $=$ $\left\{ x \in E, x \notin F_1\text{ et }x \notin F_2 \right\}$ $=$ $\left\{ x \in E, x \notin F_1 \cup F_2 \right\}$ $=$ $E$ $-$ $(F_1 \cup F_2)$.
    $U_1$ $\cap$ $U_2$ est ouvert donc $F_1 \cup F_2$ est fermé.
\end{minipage}\begin{minipage}{0.3\linewidth}
    \hspace{0.2cm}
    \begin{tikzpicture}[scale=1]
        \draw (1.55,1) circle (0.9);
        \draw (2.35,1) circle (0.9);
        \fill[dashed, color=gray!80, pattern=north west lines, pattern color=blue,even odd rule] (1.55,1) circle(0.9) (0,0) -- (0,2) -- (4,2) -- (4,0) -- cycle;
        \fill[dashed, color=gray!80, pattern=north east lines, pattern color=red,even odd rule] (2.35,1) circle(0.9) (0,0) -- (0,2) -- (4,2) -- (4,0) -- cycle;
        \draw (3.5,0.5) node[blue]{$U_1$};
        \draw (0.4,1.5) node[red]{$U_2$};
        \draw (2.5,0.5) node{$F_2$};
        \draw (1.25,0.5) node{$F_1$};
    \end{tikzpicture}
\end{minipage}
\smallbreak
Soit ($F_i$)$_{\iini}$ des fermés. Posons $U_i$ $=$ $E - F_i$. C'est un ouvert de $E$. On sait que $\underset{\iini}{\bigcup}$ $U_i$ est un ouvert.  
\smallbreak
$\underset{\iini}{\bigcup}$ $U_i$ $=$ $\left\{ x \in E, \exists \iini, x \in U_i \right\}$ $=$ $\left\{ x \in E, \forall \iini\text{ avec }x \notin F_i \right\}$.
Donc $\underset{\iini}{\bigcup}$ $U_i$ $=$ $E$ $-$ $( \underset{\iini}{\bigcap} F_i )$
On en déduit \smallbreak que $E$ $-$ $(\underset{\iini}{\bigcap} F_i)$ est un ouvert donc $(\underset{\iini}{\bigcap} F_i)$ est un fermé.

\subsection{Caractérisation des fermés par les suites.}
\newcommand{\theoreme}[1]{\parindent=0cm
\fbox{
	\begin{minipage}{0.9\textwidth}
		\underline{Théorème :} \smallbreak
		#1
	\end{minipage}
}\smallbreak}
\newcommand{\definition}[1]{\parindent=0cm
\fbox{
	\begin{minipage}{0.9\linewidth}
		\underline{Définition :} \smallbreak
		#1
	\end{minipage}
}\smallbreak}
\theoreme{Soit \evn un e.v.n, $F$ $\subset$ $E$. Il y'a équivalence entre : \smallbreak $-$ $i/$ $F$ est fermé.
\smallbreak $-$ $ii/$
$\forall $ $(x_n)_{\ninn}$ $\in$ $F$ convergeant vers $\ell$ $\in$ $E$, on a $\ell$ $\in$ $F$.}
\newcommand{\textoverline}[1]{$\overline{\mbox{#1}}$}
\newcommand{\Bb}[1]{\overline{#1}}
\newcommand{\bb}[1]{\textoverline{#1}}
\newcommand{\Ab}{\overline{A}}
\newcommand{\Ao}{\AA $\text{ }$}
\newcommand{\f}{\forall}
\newcommand{\e}{\exists}
\newcommand{\s}{\smallbreak}
\newcommand{\demo}{\parindent=0cm \smallbreak \underline{Démo :}\parindent=1cm \smallbreak}
\newcommand{\proposition}{\parindent=0cm \smallbreak \underline{Proposition :}\parindent=1cm \smallbreak}
\newcommand{\propriete}{\parindent=0cm \smallbreak \underline{Propriété :}\parindent=1cm \smallbreak}
\newcommand{\propositions}{\parindent=0cm \smallbreak \underline{Propositions :}\parindent=1cm \smallbreak}
\newcommand{\rappel}{\parindent=0cm \smallbreak \underline{Rappel :}\parindent=1cm \smallbreak}
\newcommand{\proprietes}{\parindent=0cm \smallbreak \underline{Propriétés :}\parindent=1cm \smallbreak}
\newcommand{\notation}{\parindent=0cm \smallbreak \underline{Notation :}\parindent=1cm \smallbreak}
\newcommand{\attention}{\parindent=0cm \smallbreak \underline{Attention :}\parindent=1cm \smallbreak}
\newcommand{\corollaire}{\parindent=0cm \smallbreak \underline{Corollaire :}\parindent=1cm \smallbreak}
\newcommand{\ex}{\parindent=0cm \smallbreak \underline{Ex :}\parindent=1cm \smallbreak}
\newcommand{\application}{\parindent=0cm \smallbreak \underline{Application :}\parindent=1cm \smallbreak}
\newcommand{\remarque}{\parindent=0cm \smallbreak \underline{Remarque :}\parindent=1cm \smallbreak}
\demo $i/$ $\Longrightarrow$ $ii/$. Supposons $F$ fermé. Soit $(x_n)_{\ninn}$ 
$\in$ $F$. $x_n$ $\tendplusinf$ $\ell$. Supposons que $\ell$ $\notin$ $F$ et montrons la \smallbreak contradiction. Comme $F$ est fermé, $E - F$ est ouvert et par hypothèse $\ell$ $\in$ $E - F$.
Il existe donc $r>0$, \smallbreak tel que \Bo{\ell}{r} $\subset$ $E - F$. Comme $x_n$ $\in$ $F$, $\forall$ $n$ on a $N(x_n$ $-$ $\ell)$ $\geqslant$ $r$. Par hypothèses $x_n$ $\tendplusinf$ $\ell$ i.e : \smallbreak $N(x_n$ $-$ $\ell)$ $\tendplusinf$ $0$. Donc $0$ $=$ lim $N(x_n$ $-$ $\ell)$ $\geqslant$ $r$ $>0$ absurde. Donc $\ell$ $\in$ $F$. \smallbreak
$ii/$ $\Longrightarrow$ $i/$. Montrons que $F$ est fermé ce qui équivaut à ce que  $E - F$ soit ouvert,
supposons $E - F$ n'est \smallbreak pas ouvert montre que cela contredit $i/$.
\smallbreak

($E - F$ ouvert) $\Longleftrightarrow$ ($\forall$ $\ell$ $\in$ $E - F$, $\exists$ $r>0$ avec \Bo{\ell}{r} $\subset$ $E - F$)




($\forall$ $\ell$ $\in$ $E - F$ n'est pas ouvert) $\Longleftrightarrow$ ($\exists$ $\ell$ $\in$ $E - F$, $\forall$ $r>0$, \Bo{\ell}{r} n'est pas inclus dans \smallbreak $E - F$) $\Longleftrightarrow$ ($\exists$ $\ell$ $\in$ $E - F$, $\forall$ $r>0$, \Bo{\ell}{r} $\cap$ $F$ $\neq$ $\emptyset$).
\smallbreak
Appliquons cela avec $\frac{1}{n+1}$ ($\ninn$), donc ($x_n$)$_{\ninn}$, donc \Bo{\ell}{\frac{1}{n+1}} $\cap$ $F$ $\neq$ $0$ donc il existe \smallbreak $x_n$ $\subset$ $F$ $\cap$ \Bo{\ell}{\frac{1}{n+1}} donc  $x_n$ $\in$ $F$ et \Nor{x_n - \ell} <$\frac{1}{n+1}$. \Nor{x_n - \ell} \tendplusinf \text{ }0, donc ($x_n$)$_{\ninn}$ est une suite de \smallbreak $F$ convergeant vers $\ell$ et $\ell$ $\notin$ $F$. On a donc obtenu une suite qui contredit l'hypothèse $ii/$.

\ex $-$ [$a$, $b$] est un fermé. Soit ($x_n$)$_{\ninn}$ $\in$ [$a$, $b$] convergeant vers $\ell$ $\in$ \R. On doit monter qu'en fait $\ell$ $\in$ [$a$, $b$]. Par \smallbreak hypothése on a $a$ $\leqslant$ $x_n$ $\leqslant$ $b$. En passant à la limite, on obtient $a$ $\leqslant$ $\ell$ $\leqslant$ $b$. De la même manière, [$a$, $+\infty$[, \smallbreak ]$-\infty$, $b$] sont des fermés. 
\smallbreak
$-$ [$0$, $1$[ n'est pas fermé. Il suffit des vérifier que $ii/$ du théorème n'est pas satisfaite, donc qu'il existe \smallbreak ($x_n$)$_{\ninn}$ $\in$ [$0$, $1$[ qui converge vers $\ell$ $\notin$ [$0$, $1$[.
\smallbreak
Prenons $x_n$ $=$ $1 - \frac{1}{n+1}$ $\in$ [$0$, $1$[ et $x_n$ $\longrightarrow$ $1$ $\notin$ [$0$, $1$[.
\subsection{Adhérence d'un ensemble}
\definition{Soit $A$ un s.e.v d'un e.v.n \evn. On appelle adhérence de $A$, et on note $\overline{A}$, l'ensemble $\left\{ \ell\text{ } \in\text{ } E,\text{ } \exists\text{ } (x_n)_{\ninn}\text{ } \in A\text{ avec }x_n \longrightarrow \ell \right\}$}
\remarque
$A$ $\subset$ $\overline{A}$ car si $a$ $\in$ $\Ab$, on peut écrire $a$ $=$ lim $x_n$ avec $x_n$ $=$ $a$, $\f$ $\ninn$.

\propositions $i/$ $\Ab$ est fermé \smallbreak
$ii/$ $A$ est fermé $\Longleftrightarrow$ $A$ $=$ $\Ab$
\demo $i/$ semblable à la démonstration du théorème.
\s
$ii/$ $A$ $=$ $\Ab$ $\Longrightarrow$ $A$ fermé,  découle de $i/$.
\s $A$ fermé $\Longrightarrow$ $A$ $=$ $\Ab$ car: On sait qu'on a toujours $A$ $\subset$ $\overline{A}$, il reste à voir que si $A$ est fermé, on a aussi $\Ab$ $\subset$ $A$ \s Or si $\ell$ $\in$ $\Ab$, $\e$ $(x_n)_{\ninn}$ suite de $A$ avec $x_n$ $\longrightarrow$ $\ell$.
Mais si $A$ est fermé on sait que la limite d'une telle suite \s est dans $A$ donc $\ell$ $\in$ $A$.
\ex Soit $r>0$, Alors $\underset{\text{adhérence de la boule ouverte}}{\underbrace{\text{\bb{\Bo{0}{r}}}}}$ $=$ $\underset{\text{Boule fermée}}{\underbrace{\text{\Bf{0}{r}}}}$
\demo
On sait que \Bo{0}{r} $\subset$ \Bf{0}{r} donc \bb{\Bo{0}{r}} $\subset$ \bb{\Bf{0}{r}} $=$ \Bf{0}{r} car \Bf{0}{r} est fermée. \s Il reste à voir que \Bf{0}{r} $\subset$ \bb{\Bo{0}{r}}: Soit $\ell$ avec \Nor{\ell} $=$ $r$, et soit ($t_n$)$_{\ninn}$ suite de [$0$, $1$[ $t_n$ $\longrightarrow$ $1$. \s Alors $x_n$ $=$ $t_n \ell$ \tendplusinf $ \ell$ et $N(x_n)$ $=$ $t_n N(\ell)$ $=$ $t_n r$ $<r$ donc $x_n$ $\in$ \Bo{0}{r} donc $\ell$ $\in$ \Bo{0}{r}.
\definition{Soient $A$ $\subset$ $E$, $a$ $\in$ $\Ab$ $\Longleftrightarrow$ il existe une suite $(x_n)_ {\ninn}$ de A avec $x_n$ $\tendplusinf$ $a$}
\proposition Soit $A$ $\in$ $E$ il y’a équivalence entre : \s
$i/$ $a$ $\in$ $\Ab$ \s
$ii/$ $\f$ $r>0$, \Bo{a}{r} $\cap$ $A$ $\neq$ $\emptyset$ \s
\demo
$i/$ $\Longrightarrow$ $ii/$ Soit $a$ $\in$ $\Ab$; il existe ($x_n$)$_{\ninn}$ de $A$ avec $x_n$ $\longrightarrow$ $a$.
Soit $r>0$. Comme $N(x_n$ $-$ $a)$ $\tendplusinf$ $0$ \s $\e$ $n$ avec $N(x_n$ $-$ $a)$. \s
$ii/$ $\Longrightarrow$ $i/$ appliquons $ii/$ avec $r$ $=$ $\frac{1}{n+1}$  $\ninn$. Comme $A$ $\cap$ \Bo{a}{\frac{1}{n+1}} $\neq$ $0$ $\e$ $x_n$ $\in$ $A$ avec $N(x_n$ $-$ $a)$ $<\frac{1}{1+n}$ \s donc $x_n$ $\longrightarrow$ $a$.
\subsection{Limite de fonctions}
\rappel Soient $I$ intervalle de \R, $A$ $\in$ $I$, $f$: $I$ $-$ $\left\{a \right\}$ $\longrightarrow$ \R. On dit que $f$ admet $\ell$ $\in$ \R, pour limite, lorsque \s $x$ tend vers $a$ $\Longrightarrow$ $\f$ $\varepsilon$ $>0$, $\e$ $\eta$ $>0$ et $\f$ $x$ $\in$ $I$ $-$ $\left\{a \right\}$, |$x-a$| $<\eta $ $\Longrightarrow$ |$f(x)$ $-$ $\ell$| $<\varepsilon$.
$\Longleftrightarrow$ $\f$ $\varepsilon$ $>0$ $\e$ $\eta >0$ \s et $\f$ $x$ $\in$ $(I$ $-$ $\left\{a \right\})$ $\cap$ ]$a$ $-$ $\eta$, $a$ $+$ $\eta$[ on a $f(x)$ $\in$ ]$\ell$ $-$ $\varepsilon$, $\ell$ $+$ $\varepsilon$[.
\definition{Soient $(E$, $N)$, $(E’$, $N’)$ deux e.v.n. $A$ $\subset$ $E$, $a$ $\in$ $\Ab$. Soit $f$ $\longrightarrow$ $E’$. On dit que $f$ admet une limite $\ell$ $\in$ $E'$ lorsque $x$ tend vers $a$ en restant dans $A$. $\Longleftrightarrow$ \s 
$\f$ $\varepsilon$ $>0$, $\e$ $\eta$ $>0$, $\f$ $x$ $\in$ $A$, $N(x$ $-$ $a)$ $<\eta$ $\Longrightarrow$ $N'(f(x)$ $-$ $\ell)<\varepsilon$.}
\s
$\Longleftrightarrow$ 
$\f$ $\varepsilon$ $>0$, $\e$ $\eta$ $>0$, $\f$ $x$ $\in$ $A$ $\cap$ $B_E(a$, $\eta)$, on a $f(x)$ $\in$ $B_{E'}(\ell$, $\varepsilon)$
\remarque Si la limite existe, alors elle est unique.
\demo Supposons $\ell$, $\ell'$ $\in$ $E'$ sont limites de $f$ lorsque $x$ $\longrightarrow$ $a$, $x$ $\in$ $A$. \s
$\f$ $\varepsilon$ $>0$, $\e$ $\eta$ $>0$, $\f$ $x$ $\in$ $A$, $N(x$ $-$ $a)$ $<\eta$ $\Longrightarrow$ $N'(f(x)$ $-$ $\ell)<\varepsilon$. \s
$\f$ $\varepsilon$ $>0$, $\e$ $\eta'$ $>0$, $\f$ $x$ $\in$ $A$, $N(x$ $-$ $a)$ $<\eta$ $\Longrightarrow$ $N'(f(x)$ $-$ $\ell')<\varepsilon$. \s
Posons $\eta''$ = Min [$\eta$, $\eta’$] $>0$. Comme $a$ $\in$ $\Ab$, \Bo{a}{\eta''} $\cap$ $A$ $\neq$ $\emptyset$. Soit $x$ $A$ $\cap$ $\in$ \Bo{a}{\eta''} on aura donc à la \s fois \Norp{f(x) - \ell} $<\varepsilon$ et \Norp{f(x) - \ell’} $<\varepsilon$ \s
Alors \Norp{\ell - \ell’} = \Norp{(\ell -f(x)) + (f(x) -\ell’)} $\leqslant$ \Norp{\ell -f(x)} + \Norp{(f(x) -\ell’} $< \varepsilon$ + $\varepsilon$ = $2\varepsilon$. On a \s prouvé que $\f$ $\varepsilon >0$, $0$ $\leqslant$ \Norp{\ell - \ell’} $<2\varepsilon$ d'où \Norp{\ell - \ell'} = 0. On notera $\ell$ = $\underset{x \in A}{\underset{x \rightarrow a}{\text{lim}}}f(x)$. \s
\ex $E$ = \R, Soient $I$ un intervalle de \R, $a$ $\in$ $I$, en posant $A$ = $I-\left\{ a \right\}$, on retrouve la définition usuelle de la \s limite. On a $a$ $\in$ $\Ab$. $A$ $=$  ]$a$, $b$] $b>a$. Alors $\underset{x \in A}{\underset{x \rightarrow a}{\text{lim}}}f(x)$ est la limite à droite.
\notation Lorsque $U$ est un ouvert, que $a$ $\in$ $U$, et que $A$ $=$ $U-\left\{ a \right\}$ on écrit $\underset{x \rightarrow a}{\text{lim}}f(x)$ à la place de $\underset{x \in A}{\underset{x \rightarrow a}{\text{lim}}}f(x)$.
\s
\newcommand{\ace}{$A$ $\subset$ $E$}
\proposition Soient \ace, $a$ $\in$ $\Ab$, $f:$ $A$ $\longrightarrow$ $E'$, il y'a équivalence entre : \s
$i/$ $\underset{x \in A}{\underset{x \rightarrow a}{\text{lim}}}f(x)$ existe et vaut $\ell$ $\in$ $E'$ \s
$ii/$ $\f$ $(x_n)_{\ninn}$ de $A$ convergeant vers $a$, la suite $f(x_n)_{\ninn}$ converge vers $\ell$.
\proprietes
$i/$ Si $f$ admet $\ell$ pour limite lorsque $x$ $\longrightarrow$ $a$, $x$ $\in$ $A$ alors $\ell$ $\in$ $\overline{f(A)}$. \s
$ii/$ Si $g:$ $A$ $\longrightarrow$ $E'$, $\ell$ $=$  $\underset{x \in A}{\underset{x \rightarrow a}{\text{lim}}}f(x)$ et $\ell'$ $=$  $\underset{x \in A}{\underset{x \rightarrow a}{\text{lim}}}g(x)$ existent, alors $\underset{x \in A}{\underset{x \rightarrow a}{\text{lim}}}(f + g)(x)$ existe et vaut $\ell + \ell'$ \s
$iii/$ Si $f:$ $A$ $\longrightarrow$ $E'$, $g:$ $A$ $\longrightarrow$ \R, et si  $\underset{x \in A}{\underset{x \rightarrow a}{\text{lim}}}f(x)$ $=$ $\ell$ $\in$ $E'$ existe et $\underset{x \in A}{\underset{x \rightarrow a}{\text{lim}}}g(x)$ $=$ $\lambda$ $\in$ \R $\text{ }$existe, alors \s $\underset{x \in A}{\underset{x \rightarrow a}{\text{lim}}}(f(x)g(x))$ existe et vaut $\lambda \ell$.  \s
$iv/$ Soient $E$, $E'$, $E''$ trois e.v.n, \ace, $B$ $\subset$ $E'$, $f:$ $A$ $\longrightarrow$ $E'$, $g:$ $B$ $\longrightarrow$ $E''$. Supposons $f(A)$ $\subset$ $B$, que $\ell$ $=$  \s $\underset{x \in A}{\underset{x \rightarrow a}{\text{lim}}}f(x)$ existe, on sait alors que $\ell$ $\in$ $\overline{f(A)}$ $\subset$ $\overline{B}$. Supposons $\ell'$ $=$  $\underset{y \in B}{\underset{y \rightarrow \ell}{\text{lim}}}g(y)$ existe. Alors $\underset{x \in A}{\underset{x \rightarrow a}{\text{lim}}}(g \circ f(x))$ existe \s  et vaut $\ell'$.\s
$v/$ Soit $f:$ $A$ $\longrightarrow$ \R$^p$, $x$ $\mapsto$ $f(x)$ $=$ $\begin{bmatrix}
	f_1(x) \\
	\vdots \\
	f(x_p)
\end{bmatrix}$ où $f_j:$ $A$ $\longrightarrow$ \R, alors $\underset{x \in A}{\underset{x \rightarrow a}{\text{lim}}}f(x)$ $=$ $\ell$ $=$ $\begin{bmatrix}
	\ell_1 \\
	\vdots \\
	\ell_p
\end{bmatrix}$ existe $\Longleftrightarrow$ \s $\f$ $j$ $\in$ $\left\{ 1, ..., p \right\}$ $\underset{x \in A}{\underset{x \rightarrow a}{\text{lim}}}f_j(x)$ existe et vaut $\ell_j$. \s
\demo
$i/$ Soit $(x_n)_{\ninn}$ suite de $A$, $x_n$ $\longrightarrow$ $a$, alors $f(x_n)$ $=$ $y_n$ $\longrightarrow$ $\ell$ et $y_n$ $\in$ $f(a)$, donc $\ell$ $\in$ $\overline{f(A)}$. \s
$ii/$ Si $x_n$ $\longrightarrow$ $a$ on a $f(x_n)$ $\longrightarrow$ $\ell$, $g(x_n)$ $\longrightarrow$ $\ell'$ donc $f(x_n)$ $+$ $g(x_n)$ $\longrightarrow$ $\ell$ $+$ $\ell'$. \s
$iv/$ Pour toute suite $(x_n)_{\ninn}$ de $A$, $x_n$ $\longrightarrow$ $a$, on sait que $y_n$ $=$ $f(x_n)$ $\longrightarrow$ $\ell$. Mais $(y_n)_{\ninn}$ est une suite de \s $f(a)$ $\subset$ $B$ qui converge vers $\ell$, donc $g(y_n)$ $\longrightarrow$ $\ell'$.  Donc pour toute suite $(x_n)_{\ninn}$ de $A$, $x_n$ $\longrightarrow$ $a$,\s $g \circ f(x_n)$ $\longrightarrow$ $\ell'$. \s

Soit $f:$ $A$ $\longrightarrow$ $E'$, $a$ $\in$ $A$. Alors $f$ est continue en $a$ $\Longleftrightarrow$ $\underset{x \in A}{\underset{x \rightarrow a}{\text{lim}}}f(x)$ existe et vaut $f(a)$.

\subsection{Voisinages, Intérieur d'un ensemble}
\definition{ Soit \evn un e.v.n, $a$ $\in$ $E$. On dit que $V$ $\subset$ $E$ est voisinage de $a$ de $E$. $\Longleftrightarrow$ $\e$ $r>0$ tel que \Bo{a}{r} $\subset$ $V$.}
\ex 
\begin{minipage}{0.3\linewidth}
	\begin{picture}(5.5,0.55)(0,0.15) 
	\textcolor{orange}{	
		\put(0.1,0.25){\line(1,0){0.7}}
		\put(0.1,0.16){\footnotesize{]}}
		\put(0.7,0.16){\footnotesize{[}}
		\put(-0.15,0.5){\footnotesize{$-r$}}
		\put(0.65,0.5){\footnotesize{$r$}}
	}
	\put(0.7,0.25){\line(1,0){1.3}}

	\put(0,0.25){\line(1,0){0.1}}
	\put(3.2,0.25){\line(1,0){2.3}}
	\put(0.4,0.16){[}
	\textcolor{red}{	
		\put(2,0.16){\footnotesize{]}}
		\put(2.6,0.16){\footnotesize{|}}
		\put(3.2,0.16){\footnotesize{[}}
		\put(1.55,0.5){\footnotesize{$\frac{1}{2} - r$}}
		\put(2.55,0.6){\footnotesize{$\frac{1}{2}$}} 
		\put(2.75,0.5){\footnotesize{$\frac{1}{2} + r$}}
		\put(2.0,0.25){\line(1,0){1.2}}
	}
	\put(4.8,0.16){[}
	\put(0.35,0.5){\footnotesize{$0$}}
	\put(4.75,0.5){\footnotesize{$1$}}
	\end{picture}
\end{minipage}\begin{minipage}{0.5\linewidth}
	$E$ $=$ \R. $V$ $=$ [$0$, $1$[, ici $V$ est voisinage de $1/2$, car $\e$ $r>0$ tel que \Bo{\frac{1}{2}}{r} $=$ $\left] \frac{1}{2}-r, \frac{1}{2}+r \right[$ $\subset$ $V$. Par contre $V$ n'est pas voisinage de $0$ car $\nexists$ $r>0$ avec $\left] -r, r \right[$ $\subset$ $V$.
\end{minipage} \s
 
\begin{minipage}{0.3\linewidth}
	\begin{tikzpicture}[scale=1.5]
	    \draw [->] (0,-0.1) -- (0,0.5);
	    \draw [->] (-0.1,0) -- (1.5,0);
    	\draw (0,0.6) node{\footnotesize{Im($z$)}};
	    \draw (1.8,0) node{\footnotesize{Re($z$)}};
    	\draw (0.1,0.1) node{\footnotesize{$0$}};
    	\draw (1,0.15) node{\footnotesize{$1$}};
    	\draw (1,0) node{\tiny{$|$}};
    	\draw [dashed, color=gray!80, pattern=north west lines, pattern color=blue] (0.5,0) circle (0.2);
	\end{tikzpicture}
\end{minipage} \begin{minipage}{0.6\linewidth}
	$E$ $=$ \C, avec $\f$ $z$ $\in$ \C, \Nor{z} $=$ |$z$|. Soit $A$ $=$ $\left[ 0, 1 \right[$ Alors $\left[ 0, 1 		\right[$ n'est pas voisinage de $1/2$ dans \C, car \Bo{\frac{1}{2}}{r} $=$ $\left\{  z \in \mathbb{C}, |z - \frac{1}{2}<r  \right\}$, or $\f$ $r>0$, \Bo{\frac{1}{2}}{r} non inclus dans $A$.
\end{minipage}
\proprietes
$i/$ Si $V$ est un voisinage de $a$ et $V$ $\subset$ $W$, alors $W$ est voisinage de $a$. \s
$ii/$ Si $V_1$, $V_2$ sont voisinages de $a$, alors $V_1$ $\cap$ $V_2$ est voisinage de $a$. \s
\demo
\begin{minipage}{0.2\linewidth}
	\begin{tikzpicture}[scale=0.75]
		\draw (0,0) node{$\bullet$};
		\draw (0,0) node[above]{\footnotesize{$a$}};
		\draw (0,0) node[below]{\footnotesize{\Bo{a}{r}}};
		\draw [red](-0.75,0.75) node{$V$};
		\draw [purple](-1.25,1.25) node{$W$};
		\draw [dashed, color=gray!80, pattern=north west lines, pattern color=orange] (0,0) circle (0.5);
		\draw [color=purple!80] (0,0) circle (1.5);
		\draw [dashed, color=gray!80, pattern=north west lines, pattern color=orange] (0,0) circle (0.5);
		\draw [red](-1,0) ..controls +(0.5,1.5) and +(0.5,2).. (1,-0.5);
		\draw [red](-1,0) ..controls +(-0.25,-1) and +(-0.25,1).. (0,-1);
		\draw [red](1,-0.5) ..controls +(-0.125,-0.5) and +(0.125,-0.5).. (0,-1);
	\end{tikzpicture}
\end{minipage} \begin{minipage}{0.7\linewidth}
	$i/$ Soit $V$ un voisinage de $a$, $\e$ $r>0$ avec \Bo{a}{r} $\subset$ $V$, comme $V$ $\subset$ $W$, alors \Bo{a}{r} $\subset$ $W$. Donc $W$ est un voisinage de $a$.
\end{minipage} 
\parindent=0.cm \s
\begin{minipage}{0.2585\linewidth}
	\begin{tikzpicture}[scale=1]
		\draw (0,0) node{$\bullet$};
		\draw (0,0) node[above]{\footnotesize{$a$}};
		\draw [color=purple](0,0) node[below]{\footnotesize{\Bo{a}{r_1}}};
		\draw [red](0,0) node[above right]{\footnotesize{\Bo{a}{r_2}}};
		\draw [red](0.5,0.5) node{$V_2$};
		\draw [purple](-1.3,0.5) node{$V_1$};
		\draw [color=purple!80] (-1.5,0.9) ..controls +(2,0.25) and +(2,0).. (-1,-1);
		\draw [color=purple!80] (-1.5,0.9) ..controls +(-1,-0.125) and +(-0.3,0.1).. (-2,-0.5);
		\draw [color=purple!80] (-1,-1) ..controls +(-0.6,0) and +(0.6,-0.2).. (-2,-0.5);
		\draw [dashed, color=gray!80, pattern=north west lines, pattern color=red!60] (0,0) circle (0.5);
		\draw [dashed, color=gray!80, pattern=north east lines, pattern color=purple] (0,0) circle (0.2);
		\draw [red](0,-1) ..controls +(-1.5,0.5) and +(-2,0.5).. (0.5,1);
		\draw [red](0,-1) ..controls +(1,-0.25) and +(-1,-0.25).. (1,0);
		\draw [red](0.5,1) ..controls +(0.5,-0.125) and +(0.5,0.125).. (1,0);
	\end{tikzpicture}
\end{minipage} \begin{minipage}{0.6415\linewidth}
	$ii/$ Soit $r$ $=$ Min[$r_1$, $r_2$]$>0$, alors \Bo{a}{r} $\subset$ $V_1$ $\cap$ $V_2$. Donc $V_1$ $\cap$ $V_2$ est un voisinage de $a$.
\end{minipage}
\proposition
Il y a équivalence entre : \s
$i/$ $U$ $\subset$ $E$ est un ouvert, \s
$ii/$ $\f$ $a$ $\in$ $U$, $U$ est voisinage de $a$.
\demo
$i/$ $\Longleftrightarrow$ $\f$ $a$ $\in$ $U$, $\e$ $r>0$ avec \Bo{a}{r} $\subset$ $U$ $\Longleftrightarrow$ $\f$ $a$ $\in$ $U$, $U$ est voisinage de $a$ $\Longleftrightarrow$ $ii/$.
\proposition
On peut remplacer "boules" par "voisinages" dans les définitions de  la limite, et de la continuité:  \s Si $\ell$ $=$ $\underset{x \in A}{\underset{x \rightarrow a}{\text{lim}}}f(x)$ $\Longleftrightarrow$ $(1)$:
$\f$ $\varepsilon$ $>0$, $\e$ $\eta$ $>0$, $\f$ $x$ $\in$ $A$ $\cap$ $B_E(a$, $\eta)$, on a $f(x)$ $\in$ $B_{E'}(\ell$, $\varepsilon)$. \s
Alors $(1)$ $\Longleftrightarrow$ $(2)$: $\f$ $V$ voisinage de $\ell$, $\e$ $U$ voisinage de $A$, $\f$ $x$ $\in$ $A$ $\cap$ $U$, on a $f(x)$ $\in$ $V$. 

\demo
Montrons que $(1)$ $\Longrightarrow$ $(2)$, Soit $V$ un voisinage de $\ell$, $\e$  $\varepsilon>0$ tel que \Bo{\ell}{\varepsilon} $\subset$ $V$. Appliquons $(1)$ avec $\varepsilon$. \s $\e$ $\eta$ $>0$ tel que $\f$ $x$ $\in$ $A$ $\cap$ $B_E(a$, $\eta)$, $f(x)$ $\in$ $B_{E'}(\ell$, $\varepsilon)$ $\subset$ $V$. Posons $U$ $=$ $B_E(a$, $\eta)$, c'est un voisinage qui \s vérifie $(2)$. \s
Montrons que $(2)$ $\Longrightarrow$ $(1)$, soit $\varepsilon >0$, alors $B(\ell$, $\varepsilon)$ est un voisinage de $\ell$. On peut appliquer $(2)$ à \s $V$ $=$ \Bo{\ell}{\varepsilon}. 
$\e$ $U$ voisinage de $a$ avec $x$ $\in$ $A$ $\cap$ $U$ $\Longrightarrow$ $f(x)$ $\in$ $V$ $=$ \Bo{\ell}{\varepsilon} comme $U$ est voisinage de $a$, \s $\e$ $\eta>0$ tel que \Bo{a}{\eta} $\subset$ $U$. Donc $\f$ $x$ $\in$ $A$ $\cap$ $B_E(a$, $\eta)$, on a $f(x)$ $\in$ \Bo{\ell}{\varepsilon} donc $(1)$ est vrai.
\proposition
Soit $A$ $\in$ $E$. Alors ($a$ $\in$ $\Ab$) $\Longleftrightarrow$ ($\f$ $V$ voisinage de $a$, $V$ $\cap$ $A$ $\neq$ $\emptyset$).
\demo
\fbox{$\Longrightarrow$} Soit $a$ $\in$ $\Ab$. Soit $V$ voisinage de $a$. $\e \text{ }r>0$ tel que \Bo{a}{r} $\subset$ $V$. Comme $a$ $\in$ $\Ab$, on a \Bo{a}{r} $\subset$ $A$ $\neq$ $\emptyset$, \s donc $V$ $\cap$ $A$ $\neq$ $\emptyset$.
\fbox{$\Longleftarrow$} Comme \Bo{a}{r} est voisinage de $a$, $A$ $\cap$ $B_E(a$, $\eta)$ $\neq$ $\emptyset$, $\f$ $x>0$, donc $a$ $\in$ $\Ab$.
\corollaire
Soit \ace, notons $(F_i)_{\iini}$ la famille de tous les fermés vérifiant $A$ $\subset$ $F_i$. Alors $\Ab$ $=$ $\underset{\iini}{\bigcap}F_i$.
\demo Puisque $A$ $\subset$ $F_i$, on a $\Ab$ $\subset$ $\overline{F_i}$ $=$ $F_i$ (car $F_i$ est fermé). On a donc $\Ab$ $\subset$ $\underset{\iini}{\bigcap}F_i$. \s
Montrons que $\underset{\iini}{\bigcap}F_i$ $\subset$ $\Ab$ ou encore $E - \Ab$ $\subset$ $E - (F_i)_{\iini}$.
Soit $a$ $\in$ $E - \Ab$, $a$ $\notin$ $\Ab$ donc $\e$ $r>0$ tel que \s  $\cap$ $A$ $=$ $\emptyset$. Soit $G$ $=$ $E -$\Bo{a}{r}, alors $G$ est un fermé et $A$ $\subset$ $G$. Donc $G$ est un élément de la famille $F_i$, \s $\e$ $i_0 \in I$ avec $G$ $=$ $F_{i_{0}}$. De plus $a$ $\notin$ $G$ donc $E$ $-$ $G$ $=$ $E$ $-$ $F_{i_{0}}$ $\subset$ $E$ $-$  $\underset{\iini}{\bigcap}F_i$.
\subsection{Intérieur d'un ensemble}
\definition{Soient \evn un e.v.n, \ace, on dit que $a$ est dans l'intérieur de $A$ $\Longleftrightarrow$ $\e$ $r>0$ tel que \Bo{a}{r} $\subset$ $A$. On note \Ao l'ensemble des points intérieurs de $A$.}
\ex $E$ $=$ \R, $A$ $=$ $\left[ 0, 1 \right[$
Alors \Ao $=$ $\left] 0, 1 \right[$ puisque si $x$ $\in$ $\left] 0, 1 \right[$, $\e$ $r>0$ tel que $\left] x - r,\text{ } x + r \right[$
$\subset$ $A$.\s  Par contre $0$ n'est pas intérieur à $A$. \s
$E$ $=$ $\Rd$, muni de l'une des normes ||..||$_1$, ||..||$_2$, ||..||$_{\infty}$, $A$ $=$ $\left[ 0, 1 \right[$ $\times$ $\left\{ 0 \right\}$. Alors \Ao $=$ $\emptyset$.\s $\f$ $M$ $=$ $(a$, $0)$ $\in$ $A$ avec $0$ $\leqslant$ $a$ $<1$, $\f$ $r>0$, \Bo{M}{r} $=$ $\left\{ (x, \text{ }y) \in \Rd, \text{ }x^2 + y^2< r^2 \right\}$ non inclus $A$.
\remarque
La définition équivaut à $a$ $\in$ \Ao $\Longleftrightarrow$ $\e$ $V$ voisinage de $a$ avec $V$ $\subset$ $A$.
\proprietes
$i/$ $A$ $\subset$ $B$ $\Longrightarrow$ \Ao $\subset$ \r B. \s
$ii/$ $E$ $-$ \Ao $=$ $\overline{E - A}$.
\demo
$ii/$ $x$ $\in$ $E$ $-$ \Ao $\Longleftrightarrow$ $x$ $\notin$ \Ao $\Longleftrightarrow$ $\f$ $V$ voisinage de $x$, $V$ n'est pas inclus de $A$. $\Longleftrightarrow$ $\f$ $V$ voisinage de $x$, \s $V$ $\cap$ ($E$ $-$ $A$) $\neq$ $\emptyset$ $\Longleftrightarrow$  $x$ $\in$ $\overline{E - A}$.
\corollaire
$i/$ \Ao ouvert et ($A$ est ouvert) $\Longleftrightarrow$ ($A$ $=$ \AA) \s
$ii/$ Soit $(U_i)_{\iini}$ la famille de tous les ouverts inclus dans $A$. Alors \Ao $=$ $\underset{\iini}{\bigcup}U_i$.
\demo $i/$ Soit $a$ $\in$ \Ao: $\e$ $r>0$ tel que \Bo{a}{r} $\subset$ $A$. Si $b$ $\in$ \Bo{a}{r - ||a - b||} $\subset$ \Bo{a}{r} $\subset$ $A$, donc $b$ $\in$ \AA.\s Donc on aura trouvé $r>0$ tel que \Bo{a}{r} $\subset$ \AA. Donc \Ao est ouvert. 
Si $A$ $=$ \AA, alors $A$ est ouvert. \s Réciproquement, si $A$ est ouvert et $a$ $\in$ $A$, $\e$ $V$ voisinage de $a$ avec $V$ $\subset$ $A$, donc $a$ $\in$ \AA. Donc $A$ $=$ \AA. \s
$ii/$ Comme $U_i$ $\subset$ $A$ donc $V$ $=$ $\underset{\iini}{\bigcup}U_i$ $\subset$ $A$. Or, $V$ est ouvert, donc $V$ $=$ \r V, comme $V$ $\subset$ $A$ $\Longrightarrow$ \r V $\subset$ \Ao on a \s donc $V$ $\subset$ \AA. \s
Réciproquement, si $a$ $\in$ \AA, $\exists$ $r>0$ tel que \Bo{a}{r} $\subset$ $A$. Donc \Bo{a}{r} est un ouvert inclus dans $A$,  $\e$ $i_0$ $\in$ $I$ \s tel que \Bo{a}{r} $=$ $U_{i_{0}}$. Donc $a$ $\in$ \Bo{a}{r} $\subset$ $\underset{\iini}{\bigcup}U_i$ $=$ $V$. Donc \Ao $\subset$ $V$.

\section{Compacité}
\parindent=0cm
Soit $(x_n)_{\ninn}$ une suite d'un e.v.n $E$. Une suite extraite (ou sous-suite) de $(x_n)_{\ninn}$ est une suite de la forme\s $(x_{n_k})_{\kinn}$ où $k$ $\mapsto$ $n_k$, \Nn $\longrightarrow$ \NN, est strictement croissante.
\ex $n_k$ $=$ $2k$, $(x_{n_k})_{\kinn}$ $=$ $(x_{2k})_{\kinn}$.
\remarque
$k$ $\mapsto$ $n_k$, \Nn $\longrightarrow$ \NN, strictement croissante $\Longleftrightarrow$ $\f$ $k$, $n_{k+1} > n_k$ $\Longleftrightarrow$ $n_{k+1} \geqslant n_k + 1$. (car $n_k$ $\in$ \NN, $\f$ $k$). \s En particulier $n_{k+1} \geqslant n_k + 1 \geqslant n_{k-1}+2 \geqslant$ ... $\geqslant \underset{\rightarrow +\infty \text{ si } k \rightarrow +\infty}{\underbrace{n_0 + (k+1)}}$. Donc $n_k$ $\longrightarrow$ $+\infty$ si $k$ $\longrightarrow$ $+\infty$.
\definition{Soient \evn un e.v.n, \ace. On dit que $A$ est compact $\Longleftrightarrow$ $\f$ $(x_n)_{\ninn}$ de $A$, $\e$ une sous-suite $(x_{n_k})_{\kinn}$ qui converge vers une limite $\ell \in A$. $(\ast)$}
\parindent=0cm
\underline{Définition équivalente} \parindent=1cm \smallbreak
$A$ est compact $\Longleftrightarrow$




























\end{document}
