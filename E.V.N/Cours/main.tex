\documentclass{article}
\usepackage{calc}
\usepackage[utf8]{inputenc}
\usepackage{amsmath}
\usepackage{amsthm}
\usepackage{amssymb}
\usepackage{amsfonts}
\usepackage[francais]{babel}
\usepackage{fancyvrb}
\usepackage{hyperref}
\usepackage{mathtools}
\usepackage{tikz,tkz-tab} 
\usepackage[T1]{fontenc}  
\usetikzlibrary{patterns}
\usepackage{multicol}
\usepackage{extarrows}
\usepackage[left=2cm,right=2cm,top=2cm,bottom=2cm]{geometry}
\title{Espaces Vectoriels Normés \smallbreak Fonctions à plusieurs variables}
\date{Lundi 15 janvier 2018}


\begin {document}
\newcommand{\tendplusinf}{\text{$\xlongrightarrow[n \rightarrow  + \infty]{\text{ }}$}}
\newcommand{\liminfty}{$\underset{n \rightarrow  + \infty}{\text{lim}}$}


\newcommand{\Bbarre}{$\overset{\text{\footnotesize{ }}{\bf \_}}{B}$}
\newcommand{\Bouleun}{\begin{tikzpicture}[scale=0.65]
        \draw (0,0) node[above]{\footnotesize{$a$}};
        \draw (0,0) node{\tiny{$\bullet$}};
        \draw[dashed, color=gray!80, pattern=north west lines, pattern color=blue] (0,0) circle (1);
        \draw (0,0) -- (1,0);
        \draw (1,0) node{\tiny{$\bullet$}};
     \draw (0.5,0) node[above]{\footnotesize{$r$}};
    \end{tikzpicture}
}

\newcommand{\Bouledeux}{
\begin{tikzpicture}[scale=0.65]
    \draw (0,0) node[above]{\footnotesize{$a$}};
    \draw (0,0) node{\tiny{$\bullet$}};
    \draw[blue, pattern=north west lines, pattern color=blue] (0,0) circle (1);
    \draw (0,0) -- (1,0);
    \draw (1,0) node{\tiny{$\bullet$}};
    \draw (0.5,0) node[above]{\footnotesize{$r$}};
\end{tikzpicture}
}


\newcommand{\Bouletrois}{
\begin{tikzpicture}[scale=0.65]
    \draw [->] (0,-1.5) -- (0,1.5);
    \draw [->] (-1.5,0) -- (1.5,0);
    \draw (0,1.65) node{\tiny{$x_2$}};
    \draw (1.8,0) node{\tiny{$x_1$}};
    \draw (0.2,1.2) node{\tiny{$1$}};
    \draw (1.2,0.2) node{\tiny{$1$}};
    \draw (-1.3,0.2) node{\tiny{$-1$}};
    \draw (0.3,-1.2) node{\tiny{$-1$}};
    \draw [dashed, color=gray!80, pattern=north west lines, pattern color=blue] (-1,1) -- (1,1) -- (1,-1) -- (-1,-1) -- cycle;
\end{tikzpicture}
}


\newcommand{\Boulequatre}{
\begin{tikzpicture}[scale=0.65]
    \draw [->] (0,-1.5) -- (0,1.5);
    \draw [->] (-1.5,0) -- (1.5,0);
    \draw (0,1.65) node{\tiny{$x_2$}};
    \draw (1.8,0) node{\tiny{$x_1$}};
    \draw (0.2,1.2) node{\tiny{$1$}};
    \draw (1.2,0.2) node{\tiny{$1$}};
    \draw (-1.3,0.2) node{\tiny{$-1$}};
    \draw (0.3,-1.2) node{\tiny{$-1$}};
    \draw (0,0) node[below left]{\footnotesize{$a$}};
    \draw (0,0) node{\tiny{$\bullet$}};
    \draw[dashed, color=gray!80, pattern=north west lines, pattern color=blue] (0,0) circle (1);
    \draw (0,0) -- (1,0);
    \draw (1,0) node{\tiny{$\bullet$}};
    \draw (0.5,0) node[above]{\footnotesize{$r$}};
\end{tikzpicture}
}

\newcommand{\Boulecinq}{
\begin{tikzpicture}[scale=0.65]
    \draw [->] (0,-1.5) -- (0,1.5);
    \draw [->] (-1.5,0) -- (1.5,0);
    \draw (0,1.65) node{\tiny{$x_2$}};
    \draw (1.8,0) node{\tiny{$x_1$}};
    \draw (0.2,1.2) node{\tiny{$1$}};
    \draw (1.2,0.2) node{\tiny{$1$}};
    \draw (-1.3,0.2) node{\tiny{$-1$}};
    \draw (0.3,-1.2) node{\tiny{$-1$}};
    \draw [dashed, color=gray!80, pattern=north west lines, pattern color=blue] (-1,0) -- (0,1) -- (1,0) -- (0,-1) -- cycle;
\end{tikzpicture}
}
\maketitle
\bigbreak

\parindent=0cm 
Soit $E$ un espace vectoriel.
\smallbreak
Le but de cette partie est de définir la notion de distance entre deux vecteurs de $E$.
\smallbreak
\underline{Ex:} $-$ $E = \mathbb{R}$ :
On peut définir la distance entre deux réels $a$ et $a'$ par $d(a,a') = |a - a'|$.
\smallbreak
\hspace*{0.72cm}$-$ $E = \mathbb{C}$ : On peut poser $z$, $z'$ $\in$ $\mathbb{C}$, $d(z,z') = |z - z'|$. Si on écrit $z = x + iy$ et $z' = x' + iy'$, alors \smallbreak $d(z,z') = |x$ $-$ $x'$ $+$ $i(y$ $-$ $y')| = \sqrt{(x - x')^2 + (y - y')^2}$.
\smallbreak
\hspace*{0.72cm}$-$ $E = \mathbb{R}^2$ : Si ($x$, $y$), ($x'$, $y'$) $\in$ $\mathbb{R}^2$, on peut définir $d((x$, $y$), ($x'$, $y'$))$ = \sqrt{(x - x')^2 + (y - y')^2}$.
\medbreak
\section{Normes et distances sur un espace vectoriel}
\medbreak
\fbox{
	\begin{minipage}{0.9\textwidth}
		\underline{Définition :} Soit $E$ un espace vectoriel sur $\mathbb{R}$. Une norme sur $E$ est par définition une application \smallbreak $N$ : $E$ $\rightarrow$ $\mathbb{R}$, $x$ $\mapsto$ $N(x)$ vérifiant : \smallbreak 
		$-$ $\forall$ $x \in E$,  $N(x)$ $\geqslant$ $0$, et ($N(x)$ $=$ $0$ $\Longleftrightarrow$ $x = 0$), \smallbreak
		$-$ $\forall$ $x \in E$,  $\forall$ $\lambda \in \mathbb{R}$, $N(\lambda x)$ $=$ $|\lambda|N(x)$, \smallbreak
		$-$ (Inégalité triangulaire) $\forall$ $x$, $y \in E$, $N(x$ $+$ $y)$ $\leqslant$ $N(x)$ $+$ $N(y)$.
	\end{minipage}
}
\smallbreak
\fbox{
	\begin{minipage}{0.9\linewidth}
		\underline{Définition :} Un espace vectoriel normé (e.v.n) est un couple ($E$, $N$) où $E$ est un e.v et $N$ une norme sur $E$.
	\end{minipage}
}
\smallbreak
\underline{Propriétés :}
\parindent=1cm  
\smallbreak
$-$ Soient $x_1$, $x_2$, $x_3$ $\in E$ avec ($E$, $N$) un e.v.n. Alors,\smallbreak $N(x_1$ $+$ $x_2$ $+$ $x_3)$ $=$ $N((x_1$ $+$ $x_2$) $+$ $x_3)$ $\leqslant$ $N(x_1$ $+$ $x_2$) $+$ $N(x_3)$ $\leqslant$ $N(x_1)$ $+$ $ N(x_2)$ $+$ $N(x_3)$
\smallbreak
$-$ Si $x_1$, ..., $x_p$ $\in$ $E$, $N(x_1$, ..., $x_p)$ $\leqslant$ $N(x_1)$ $+$ ... $+$ $N(x_p)$.
\smallbreak
$-$ Si $x$, $y$ $\in$ $E$, |$N(x)$ $-$ $N(y)$| $\leqslant$ $N(x$ $-$ $y)$.

\parindent=0cm
\smallbreak
\underline{Démo :} 
\smallbreak
\parindent=1cm
$N(x)$ $=$ $N((x$ $-$ $y)$ $+$ $y)$ $\leqslant$ $N(x$ $-$ $y)$ $+$ $N(y)$, $N(x)$ $-$ $N(y)$ $\leqslant$ $N(x$ $-$ $y)$. \smallbreak Aussi $N(y)$ $-$ $N(x)$ $\leqslant$ $N(y$ $-$ $x)$ $=$ $N((-1)(x$ $-$ $y))$ $=$ |$-$1|$N(x$ $-$ $y)$ $=$ $N(x$ $-$ $y)$. \smallbreak Finalement, |$N(x)$ $-$ $N(y)$| = Max$(N(x)$ $-$ $N(y)$, $N(y)$ $-$ $N(x))$ $\leqslant$ $N(x$ $-$ $y)$.
\parindent=0cm
\smallbreak

\underline{Ex :}
\parindent=1cm
\smallbreak
$-$ $E = \mathbb{R}$ : Posons $N(x)$ = |$x$| (Valeur absolue).\smallbreak $N$ est une norme sur $\mathbb{R}$, car |$x$| $\geqslant$ $0$, (|$x$| $=$ $0$ $\Longleftrightarrow$ $x = 0$), \smallbreak $\forall$ $\lambda \in \mathbb{R}$, |$\lambda x$| $=$ |$\lambda$||$x$|, et $\forall$ $x$, $y \in E$, |$x$ $+$ $y$| $\leqslant$ |$x$| $+$ |$y$|.
\smallbreak
$-$ $E = \mathbb{C}$ : Posons $N(x)$ = |$x$| (Module).\smallbreak $N$ est une norme sur $\mathbb{C}$, car |$z$| $\geqslant$ $0$, (|$z$| $=$ $0$ $\Longleftrightarrow$ $z = 0$), \smallbreak $\forall$ $\lambda \in \mathbb{C}$, |$\lambda z$| $=$ |$\lambda$||$z$|, et $\forall$ $z$, $z' \in E$, |$z$ $+$ $z'$| $\leqslant$ |$z$| $+$ |$z'$|.
\smallbreak
$-$ Espaces euclidiens :
\smallbreak Soit $E$ un e.v. Une forme bilinéaire est une application $B$ : $E$ $\times$ $E$ $\rightarrow$ $\mathbb{R}$, $(x$, $y)$ $\mapsto$ $B(x$, $y)$ telle qu'elle est \smallbreak linéaire en chacune de ses variables. \smallbreak Un produit scalaire sur un e.v $E$ est une forme bilinéaire symétrique sur $E$ définie positive au sens suivant : \smallbreak $\forall$ $x \in E$,  $B(x,x)$ $\geqslant$ $0$, et $B(x,x)$ $=$ $0$ $\Longleftrightarrow$ $x = 0$. Un espace euclidien est un e.v muni d'un produit scalaire. \smallbreak On pose $B(x$, $y)$ = $\overset{n}{\underset{j = 1}{\sum}} x_j y_j$. $B$ est un produit scalaire sur $\mathbb{R}^n$ avec ($x$, $y$ $\in$ $\mathbb{R}^n$).
\smallbreak
On pose maintenant : $N(x)$ $=$ $\sqrt{B(x, x)}$. Alors, $N$ est une norme sur $E$.\smallbreak
\underline{Démo :} \parindent=2cm \smallbreak
$i$/ Par définition, $N(x)$ $\geqslant$ $0$, et $N(x)$ $=$ $0$ $\Longleftrightarrow$ $B(x,x)$ $=$ $0$ $\Longleftrightarrow$ $x = 0$ \smallbreak
$ii$/ Si $\lambda$ $\in$ $\mathbb{R}$, $x$ $\in$ $E$, $N(\lambda x)$ $=$ $\sqrt{B(\lambda x,\lambda x)} $ $=$ $\sqrt{\lambda^2 B(x,x)}$ $=$ |$\lambda$|$\sqrt{B(x,x)}$ $=$ $|\lambda|N(x)$,\smallbreak
$iii$/ On va utiliser le lemme : $\forall$ $x$, $y \in E$, posons $p(\lambda)$ $=$ $B(x + \lambda y, x)$ $+$ $\lambda B(x + \lambda y, y)$. Si on pose \smallbreak $z$ $=$ $x$ $+$ $\lambda y$ fixé, et $u(\omega)$ $=$ $B(z, \omega)$, on a écrit que ($u(x$ $+$ $\lambda y)$ $=$ $u(x)$ $+$ $\lambda u(y)$). \smallbreak On aura $p(\lambda)$ $=$ $B(x$, $x)$ $+$ $\lambda B(y$, $x)$ $+$ $\lambda B(x$, $y)$ $+$ $\lambda^2$$B(y$, $y)$ donc $p(\lambda)$ $=$ $B(x$, $x)$ $+$ $2\lambda B(y$, $x)$ $+$\smallbreak $\lambda^2$$B(y$, $y)$ donc $\lambda$ $\mapsto$ $p(\lambda)$ est un polynôme de degré $\leqslant$ $2$ et $p(\lambda)$ $=$ $B(x + \lambda y, x + \lambda y)$ $\geqslant$ $0$. \smallbreak Or, si un polynôme de degré $\leqslant$ $2$ ne change pas de signe, son discriminant est $\leqslant$ $0$. \smallbreak ($2B(x$, $y))^2$ $-$ $4B(x$, $x)B(y$, $y)$ $\leqslant$ $0$, donc |$B(x$, $y)$| $\leqslant$ $\sqrt{B(x, x)}$ $\sqrt{B(y, y)}$. On a donc $N(x$ $+$ $y)^2$ \smallbreak $=$ $B(x$  $+$ $y$, $x$  $+$ $y)$ $=$ $B(x$, $x$  $+$ $y)$ $+$ $B(y$, $x$  $+$ $y)$ $=$ $B(x$, $x)$ $+$ $B(y$, $x)$ $+$ $B(x$, $y)$ $+$ $B(y$, $y)$ $=$\smallbreak $B(x$, $x)$ $+$ $2B(x$, $y)$ $+$ $B(y$, $y)$ $\leqslant$ $N(x)^2$ $+$ $N(x)N(y)$ $+$ $N(y)^2$. \smallbreak On a obtenu $N(x$ $+$ $y)^2$ $\leqslant$ $(N(x)$ $+$ $N(y))^2$ soit $N(x$ $+$ $y)$ $\leqslant$ $N(x)$ $+$ $N(y)$
\parindent=1cm
\smallbreak
$-$ $E = \mathbb{R}^n$ : $B(x$, $y)$ $=$ $\Sigma$ $x_i$$y_i$. La norme obtenue se note ||..||$_2$ et pour $x$ $=$ 
$
\begin{bmatrix}
	x_1 \\
	\vdots \\
	x_n
\end{bmatrix}
$
 est donnée par \smallbreak ||x||$_2$ $\overset{ def }{=}$ $($$\overset{n}{\underset{i = 1}{\sum}} x_i^2)^{\frac{1}{2}}$
\smallbreak
$-$ Autre exemple de normes : Soient $E$, $E'$ deux e.v et $\varphi$ : $E$ $\rightarrow$ $E'$ linéaire injective. \smallbreak Soit $N'$ une norme sur $E'$. Pour $x$ $\in$ $E$, posons $N(x)$ $\overset{ def }{=}$ $N'(\varphi (x))$. Alors $N$ est une norme sur $E$:
\smallbreak $i$/ $N(x)$ $\geqslant$ $0$ de plus $N(x)$ $=$ $0$ $\Longrightarrow$ $N'(\varphi(x))$ $=$ $0$ $\Longrightarrow$ $\varphi(x)$ $=$ $0$ (car $N'$ est une norme) $\Longrightarrow$ $x$ $=$ $0$. (puisque \smallbreak$\varphi$ est injective).
\smallbreak
$ii$/ $N(\lambda x) = N'(\varphi(\lambda x))$ $=$ $N'(\lambda \varphi(x))$ $=$ |$\lambda$|$N'(\varphi(x))$ $=$ |$\lambda$|$N(x)$
\smallbreak
$iii$/ $N(x$ $+$ $y)$ $=$ $N'(\varphi(x$ $+$ $y))$ $=$ $N'(\varphi(x)$ $+$ $\varphi(y))$ $\leqslant$ $N'(\varphi(x))$ $+$ $N'(\varphi(y))$ $=$ $N(x)$ $+$ $N(y)$
\parindent=0cm
\subsection{Normes usuelles de $\mathbb{R}^p$}
Soit $x$ $=$ $
\begin{bmatrix}
	x_1 \\
	\vdots \\
	x_p
\end{bmatrix}
$
$\in$ $\mathbb{R}^p$, les normes usuelles de $\mathbb{R}^p$ sont définies par :

\smallbreak
\fbox{
	\begin{minipage}{0.9\textwidth}
		\center 
		||$x$||$_1$ $\overset{ def }{=}$ $\overset{p}{\underset{j = 1}{\sum}}|x_j|$,\hspace*{0.5cm}
		||$x$||$_2$ $\overset{ def }{=}$ $($$\overset{p}{\underset{j = 1}{\sum}} x_j^2)^{\frac{1}{2}}$,\hspace*{0.5cm}
		||$x$||$_{\infty}$ $\overset{ def }{=}$ $\underset{j \in \left\{ 0,\text{ }...,\text{ }p \right\} }{\text{Max}}$[$|x_j|$].\hspace*{0.5cm} 

		\flushleft
	\end{minipage}
}
\smallbreak
\underline{Propriété :} ||..||$_1$, ||..||$_2$, ||..||$_{\infty}$, sont des normes de $\mathbb{R}^p$.
\smallbreak
\underline{Démo :}
\parindent=1cm
\smallbreak
$-$ Pour ||..||$_2$ : voir plus haut.
\smallbreak
$-$ Pour ||..||$_1$ :
\smallbreak
Soit $x$ $\in E$, ||$x$||$_1$ $=$ $\overset{p}{\underset{j = 1}{\sum}}|x_j|$ $\geqslant$ $0$, et $\overset{p}{\underset{j = 1}{\sum}}|x_j|$ $=$ $0$ $\Longleftrightarrow$ $\forall$ $j$ $x_j = 0$ $\Longleftrightarrow$ $x = 0$.
\smallbreak
Soit $\lambda$ $\in$ $\mathbb{R}$, ||$\lambda x$||$_1$ $=$ $\overset{p}{\underset{j = 1}{\sum}}|\lambda x_j|$ $=$ |$\lambda$|$(\overset{p}{\underset{j = 1}{\sum}}|x_j|)$ $=$ |$\lambda$|||$x$||$_1$.
\smallbreak
Soient $x$, $y$ $\in$ $\mathbb{R}^p$, ||$x$ $+$ $y$||$_1$ $=$ $\overset{p}{\underset{j = 1}{\sum}}|x_j + y_j|$ $\leqslant$ $\overset{p}{\underset{j = 1}{\sum}}(|x_j| + |y_j|)$ $=$ $\overset{p}{\underset{j = 1}{\sum}}|x_j|$ $+$ $\overset{p}{\underset{j = 1}{\sum}}|y_j|$ $=$ ||$x$||$_1$ $+$ ||$y$||$_1$
\smallbreak
$-$ Pour ||..||$_{\infty}$ :
\smallbreak
||$x$||$_{\infty}$ $=$ 0 $\Longrightarrow$ $\underset{j \in \left\{ 1,\text{ }...,\text{ }p \right\} }{\text{Max}}$[$|x_j|$] $=$ $0$. $\Longrightarrow$ $\forall$ $j$ $x_j = 0$ $\Longleftrightarrow$ $x = 0$.
\smallbreak
Vérifions l'inégalité triangulaire : |$x_j$ + $y_j$| $\leqslant$ |$x_j$| + |$y_j$| $=$ ||$x_j$||$_{\infty}$ $+$ ||$y_j$||$_{\infty}$ donc \smallbreak $\underset{j \in \left\{ 1,\text{ }...,\text{ }p \right\} }{\text{Max}}$[|$x_j$| + |$y_j$|] $\leqslant$ ||$x_j$||$_{\infty}$ + ||$y_j$||$_{\infty}$.
\smallbreak
\parindent=0cm
\underline{Propriété :} $\forall$ $x$ $\in$ $\mathbb{R}^p$, ||$x$||$_{\infty}$ $\leqslant$ ||$x$||$_2$ $\leqslant$ ||$x$||$_1$ $\leqslant$ $p$||$x$||$_{\infty}$.
\smallbreak
\underline{Démo :} \parindent=1cm $\forall$ $j \in \left\{ 1,\text{ }...,\text{ }p \right\} $, |$x_j$| $\leqslant$ $\left(\overset{p}{\underset{j = 1}{\sum}} x_j^2\right)^{\frac{1}{2}}$ donc ||$x$||$_{\infty}$ $=$ $\underset{j \in \left\{ 1,\text{ }...,\text{ }p \right\} }{\text{Max}}$[$|x_j|$] $\leqslant$ ||$x$||$_2$.
\smallbreak
Pour montrer que ||$x$||$_2$ $\leqslant$ ||$x$||$_1$, il suffit de vérifier que ||$x$||$_2^2$ $\leqslant$ ||$x$||$_1^2$
\smallbreak
Soit $\left( \overset{p}{\underset{j = 1}{\sum}} x_j\right)^2$ $=$ $\left( \overset{p}{\underset{j = 1}{\sum}} x_j\right)$$\left(\overset{p}{\underset{j = 1}{\sum}} x_j\right)$ $=$ $\overset{p}{\underset{j = 1}{\sum}}\left( \overset{p}{\underset{i = 1}{\sum}} |x_j||x_i|\right)$ $=$ $\overset{p}{\underset{j = 1}{\sum}} x_j^2$ $+$ $\underset{i \neq j}{\overset{p}{\underset{j = 1}{\sum}} \overset{p}{\underset{i = 1}{\sum}}}$ |$x_j$||$x_i$|. Or, $\underset{i \neq j}{\overset{p}{\underset{j = 1}{\sum}} \overset{p}{\underset{i = 1}{\sum}}}$ |$x_j$||$x_i$| $\geqslant$ $0$, donc \smallbreak ||$x$||$_2^2$ $\leqslant$ ||$x$||$_1^2$, enfin ||$x$||$_2$ $\leqslant$ ||$x$||$_1$.
\smallbreak
||$x$||$_1$ $=$ $\overset{p}{\underset{j = 1}{\sum}}|x_j|$ $\leqslant$ $\overset{p}{\underset{j = 1}{\sum}}||x||_{\infty} $ $=$ ||$x$||$_{\infty}$$\overset{p}{\underset{j = 1}{\sum}} 1 $ = $p$||$x$||$_{\infty}$
\smallbreak
\parindent=0cm
\underline{Remarque :} \parindent=1cm \smallbreak Si $E$ est un e.v de dimension finie $p$, et si ($e_1$, ..., $e_p$) est une base de $E$, alors $\forall$ $x$ $\in$ $E$ s'écrit de \smallbreak manière unique $x$ $=$ $\overset{p}{\underset{j = 1}{\sum}}x_j e_j$. On peut donc définir des normes $N_1$, $N_2$, $N_{\infty}$ sur $E$ en posant \smallbreak $N_{\ell}$ $\overset{ def }{=}$ $\left| \left|
\begin{bmatrix}
	x_1 \\
	\vdots \\
	x_n
\end{bmatrix}
\right| \right|_{\ell}$, $\ell$ $\in$ $\left\{1, 2, \infty \right\}$

\subsection{Norme produit}
\parindent=0cm
Soient ($E_1$, $N_1$),($E_2$, $N_2$) deux e.v.n. Soit $E_1$ $\times$ $E_2$ $=$ $\left\{ x = (x_1, x_2),\text{ }  x_1 \in E_1,\text{ } x_2 \in E_2 \right\}$. 

\smallbreak
\fbox{
	\begin{minipage}{0.9\textwidth}
		\underline{Définition :} $N(x)$ $\overset{ def }{=}$ Max[$N_1(x_1)$, $N_2(x_2)$] est une norme sur $E$, et est appelée norme produit.
	\end{minipage}
}

\subsection{Distance associée à une norme}

\fbox{
	\begin{minipage}{0.9\textwidth}
		\underline{Définition :} Soit ($E$, $N$) un e.v.n. la distance $d(x$, $y)$ entre $x$ $\in$ $E$ et $y$ $\in$ $E$, associée à $N$ est par définition $d(x$, $y)$ $=$ $N(x$ $-$ $y)$.
	\end{minipage}
}

\smallbreak
\underline{Propriété :} La distance précédente est une application $d$ : $E$ $\times$ $E$ $\rightarrow$ $\mathbb{R}$ vérifiant :
\parindent=1cm
\smallbreak
$i$/ $\forall$ ($x$, $y$) $\in$ $E$ $\times$ $E$ $d(x$, $y)$ $\geqslant$ $0$ et ($d(x$, $y)$ $=$ $0$ $\Longleftrightarrow$ $x = y$) \smallbreak
$ii$/ (symétrie) $\forall$ $x$, $y$ $\in$ $E$, $d(x$, $y)$ $=$ $d(y$, $x)$ \smallbreak
$iii$/ (Inégalité triangulaire) $\forall$ $x$, $y$, $z$ $\in$ $E$, $d(x$, $z)$ $\leqslant$ $d(x$, $y)$ $+$ $d(y$, $z)$ 
\parindent=0cm
\smallbreak
\underline{Démo :} \parindent=1cm \smallbreak $ii$/ $d(x$, $y)$ $=$ $N(y$ $-$ $x)$ $=$ $N((-1)(x$ $-$ $y))$ $=$ |$-$1|$N(x$ $-$ $y)$ $=$ $N(x$ $-$ $y)$ $=$ $d(y$, $x)$ \smallbreak
$iii$/ $d(x$, $z)$ $=$ $N(x$ $-$ $z)$ $=$ $N((x$ $-$ $y)$ $+$ $(y$ $-$ $z))$ $\leqslant$ $N(x$ $-$ $y)$ $+$ $N(y$ $-$ $z)$ $=$ $d(x$, $y)$ $+$ $d(y$, $z)$ 
\parindent=0cm
\smallbreak
\underline{Remarque :} \parindent=1cm \smallbreak De manière générale, si $E$ est une ensemble, on définit une $d$ distance sur $E$ comme une application vérifiant \smallbreak $d$ : $E$ $\times$ $E$ $\rightarrow$ $\mathbb{R}$ vérifiant $i$/, $ii$/, $iii$/.
\parindent=0cm
\smallbreak
C'est une notion de distance plus générale de la distance associée à une norme. Si $d(x$, $y)$ $=$ $N(x$ $-$ $y)$. On peut prendre par exemple pour tout $a$ $\in$ $E$, $d(x$ $+$ $a$, $y$ $+$ $a)$ $=$ $N((x$ $+$ $a)$ $-$ $(y$ $+$ $a))$ $=$ $N(x$ $-$ $y)$ $=$ $d(x$, $y)$. Cette propriété n'est pas toujours vraie pour une distance "normale".
\smallbreak
Soit ($E$, $N$) un e.v.n. Soit $a$ $\in$ $E$.\smallbreak
\fbox{
	\begin{minipage}{0.9\textwidth}
		\underline{Définition :} \smallbreak
		$i$/ Soit $r>0$, la boule ouverte de centre $a$ et de rayon $r$ est par définition $B(a$, $r)$ $=$ $\left\{ x \text{ }\in \text{ }E,\text{ }N(x\text{ }-\text{ }a)\text{ }<\text{ }r \right\}$ $=$ $\left\{ x \text{ }\in \text{ }E,\text{ }d(x,\text{ }a)\text{ }<\text{ }r \right\}$ \smallbreak
		$ii$/ Soit $r\geqslant 0$, la boule fermée de centre $a$ et de rayon $r$ est par définition \Bbarre$(a$, $r)$ $=$ $\left\{ x \text{ }\in \text{ }E,\text{ }N(x\text{ }-\text{ }a)\text{ } \leqslant \text{ }r \right\}$ $=$ $\left\{ x \text{ }\in \text{ }E,\text{ }d(x,\text{ }a)\text{ } \leqslant \text{ }r \right\}$
	\end{minipage}
}
\smallbreak
\underline{Remarque :} \parindent=1cm \smallbreak
$-$ Soit $r>0$, alors $a$ $\in$ $B(a$, $r)$ $\subset$ \Bbarre$(a$, $r)$ \smallbreak
$-$ \Bbarre$(a$, $0)$ $=$ $\left\{ a \right\}$ \smallbreak
$-$ Si $r<r'$ $B(a$, $r)$ $\subset$ $B(a$, $r')$, \Bbarre$(a$, $r)$ $\subset$ \Bbarre$(a$, $r')$

\parindent=0cm
\smallbreak
\underline{Ex :}
\parindent=1cm
\smallbreak
$-$ $E = \mathbb{R}$ , $N(x)$ $=$ |$x$|. Soit $a$ $\in$ $\mathbb{R}$, alors $B(a$, $r)$ $=$ $\left\{ x \text{ }\in \text{ }E,\text{ }|x\text{ }-\text{ }a|\text{ }<\text{ }r \right\}$ $=$ ]$a$ $-$ $r$, $a$ $+$ $r$[,
\setlength{\unitlength}{0.8cm} 
\begin{picture}(3,0.5)(0,0.15) \put(0,0.25){\line(1,0){3}}
\put(0.4,0.16){]}
\put(2.6,0.16){[}
\put(1.5,0.16){|}
\put(0,0.6){\footnotesize{$a$ $-$ $r$}}
\put(1.45,0.6){\footnotesize{$a$}}
\put(2.2,0.6){\footnotesize{$a$ $+$ $r$}}
\put(0.5,0.16){\textcolor{blue}{$\smallsetminus \smallsetminus \smallsetminus \smallsetminus \smallsetminus$}}

\end{picture}
\smallbreak

\Bbarre$(a$, $r)$ $=$ $\left\{ x \text{ }\in \text{ }\mathbb{R},\text{ }|x\text{ }-\text{ }a|\text{ } \leqslant \text{ }r \right\}$ $=$[$a$ $-$ $r$, $a$ $+$ $r$], \begin{picture}(3,0.5)(0,0.15) \put(0,0.25){\line(1,0){3}}
\put(2.6,0.16){]}
\put(0.4,0.16){[}
\put(1.5,0.16){|}
\put(0,0.6){\footnotesize{$a$ $-$ $r$}}
\put(1.45,0.6){\footnotesize{$a$}}
\put(2.2,0.6){\footnotesize{$a$ $+$ $r$}}
\put(0.35,0.16){\textcolor{blue}{$\smallsetminus \smallsetminus \smallsetminus \smallsetminus \smallsetminus \smallsetminus$}}
\end{picture}

\smallbreak


$-$ $E = \mathbb{C}$, $N(z)$ $=$ |$z$|, pour $z$ $\in$ $\mathbb{C}$. Soient $a$, $r>0$.
\begin{multicols}{2}
\parindent=1cm
Alors $B(a$, $r)$ $=$ $\left\{ z \text{ }\in \text{ }\mathbb{C},\text{ }|z\text{ }-\text{ }a|\text{ }<\text{ }r \right\}$ c'est le 
\hspace*{1cm}disque \underline{ouvert} de centre $a$, et de rayon $r$:

\columnbreak
\Bouleun


\end{multicols}
\begin{multicols}{2}
\parindent=1cm
Alors \Bbarre$(a$, $r)$ $=$ $\left\{ z \text{ }\in \text{ }\mathbb{C},\text{ }|z\text{ }-\text{ }a|\text{ } \leqslant \text{ }r \right\}$ c'est le 
\hspace*{1cm}disque \underline{fermé} de centre $a$, et de rayon $r$:

\columnbreak
\Bouledeux



\end{multicols}
$-$ $E = \mathbb{R}^p$ muni des normes ||..||$_1$, ||..||$_2$, ||..||$_{\infty}$, on a vu que $\forall$ $x$ $\in$ $\mathbb{R}^p$, ||$x$||$_{\infty}$ $\leqslant$ ||$x$||$_2$ $\leqslant$ ||$x$||$_1$ $\leqslant$ $p$||$x$||$_{\infty}$\smallbreak Notons $B_{\ell}(a$, $r)$ $=$ $\left\{ x \text{ }\in \text{ }\mathbb{R}^p,\text{ }||x\text{ }-\text{ }a||_{\ell}\text{ } \leqslant \text{ }r \right\}$. $\ell$ $\in$ $\left\{1, 2, \infty \right\}$

\smallbreak 
Soit $r>0$, soit $x$ $\in$ $B_{\ell}(0$, $\frac{r}{p})$. Alors ||$x$||$_{\infty}$ $<$ $\frac{r}{p}$, donc ||$x$||$_1$ $\leqslant$ $p$||$x$||$_{\infty}$ $<$ $p\frac{r}{p}$ $=$ $r$ or $x$ $\in$ $B_1(0$, $r)$.
\smallbreak
On en déduit que $B_{\infty}(0$, $\frac{r}{p})$ $\subset$ $B_1(0$, $r)$
De même $B_1(0$, $r)$ $\subset$ $B_2(0$, $r)$ $\subset$ $B_{\infty}(0$, $r)$.
\bigbreak
\center 
$\forall$ $x$ $\in$ $E$, $B_{\infty}(0$, $\frac{r}{p})$ $\subset$ $B_1(0$, $r)$ $\subset$ $B_2(0$, $r)$ $\subset$ $B_{\infty}(0$, $r)$.

\flushleft
\hspace*{1cm}Cas de $p$ $=$ $2$, $r$ $=$ $1$
\smallbreak

\begin{multicols}{2}

\parindent=1.5cm
Alors $B_{\infty}(0$, $1)$ $=$ $\left\{x=(x_1,\text{ }x_2) \in \mathbb{R}^2,||x||_{\infty} \right\}$ \smallbreak $=$ $\left\{(x_1,\text{ }x_2) \in \text{ }\mathbb{R}^2,\text{Max[|}x_1\text{|, |}x_2\text{|]} <1 \right\}$

\smallbreak
$=$ $\left\{(x_1,\text{ }x_2) \in \mathbb{R}^2,|x_1| < 1 \text{ et } |x_2| < 1 \right\}$
\smallbreak
\hspace{1cm}
\columnbreak

\Bouletrois
\end{multicols}

\smallbreak

\begin{multicols}{2}

\parindent=1.5cm
Alors $B_{2}(0$, $1)$ $=$ $\left\{x=(x_1,\text{ }x_2) \in \mathbb{R}^2,||x||_{2} \right\}$ \smallbreak $=$ $\left\{(x_1,\text{ }x_2) \in \text{ }\mathbb{R}^2,\sqrt{x_{1}^2 + x_{2}^2} <1 \right\}$
\smallbreak
$=$ $\left\{(x_1,\text{ }x_2) \in \mathbb{R}^2,|x_1| < 1 \text{ et } |x_2| < 1 \right\}$ \smallbreak
$=$ $\left\{z = x_1 + ix_2) ,|z| < 1 \right\}$
\columnbreak

\Boulequatre
\end{multicols}
\parindent=1.5cm
\smallbreak
Alors $B_{\infty}(0$, $1)$ $=$ $\left\{x=(x_1,\text{ }x_2) \in \mathbb{R}^2,||(x_1,\text{ }x_2)||_{1} \right\}$ $=$ $\left\{(x_1,\text{ }x_2) \in \mathbb{R}^2,|x_1| + |x_2| < 1 \right\}$
\begin{multicols}{2}

\parindent=1.5cm
Alors $B_{1}(0$, $1)$ $\cap$ $\left\{(x_1,\text{ }x_2), x_1 \leqslant 0, x_2 \right\}$ $=$ $A$. \smallbreak $B_{1}(0$, $1)$ s'obtient par symétrie par A, par \smallbreak rapport aux axes.
\smallbreak
\hspace{1cm}
\smallbreak
\hspace{1cm}
\smallbreak
\columnbreak

\Boulecinq
\end{multicols}
\parindent=0cm
\smallbreak
\fbox{
	\begin{minipage}{0.9\textwidth}
		\underline{Définition :} \smallbreak
		On dit que deux normes $N_1$ et $N_2$ sont équivalentes ssi $\exists$ $C>0$, telle que $\forall$ $x$ $\in$ $E$,
		 $N_1(x)$ $\leqslant$ $CN_2(x)$ et  $N_2(x)$ $\leqslant$ $CN_1(x)$.
	\end{minipage}
}
\smallbreak
\underline{Remarque :} \smallbreak
On définit aussi "$N_1$ et $N_2$ sont équivalentes" par:
\parindent=1cm
\smallbreak
$\exists$ $C_1>0$, telle que $\forall$ $x$ $\in$ $E$, $N_2(x)$ $\leqslant$ $C_1 N_1(x)$ et,
$\exists$ $C_2>0$, telle que $\forall$ $x$ $\in$ $E$, $N_1(x)$ $\leqslant$ $C_2 N_2(x)$.

\parindent=1cm
\smallbreak
Cette définition est équivalente à la précédente. 

Déf 1 $\Longrightarrow$ Déf 2, c'est évident, on prend $C_1$ $=$ $C_2$ $=$ $C$. Déf 1 $\Longrightarrow$ Déf 2, en prennant $C$ $=$ Max[$C_1$, $C_2$].
\parindent=0cm

\underline{Ex :}
Sur $\mathbb{R}^p$ ||..||$_1$, ||..||$_2$, ||..||$_{\infty}$, sont des normes équivalentes (car $\forall$ $x$ $\in$ $\mathbb{R}^p$, ||$x$||$_{\infty}$ $\leqslant$ ||$x$||$_2$ $\leqslant$ ||$x$||$_1$ $\leqslant$ $p$||$x$||$_{\infty}$.)

\section{Limites et continuité}
\fbox{
	\begin{minipage}{0.9\textwidth}
		\underline{Définition :} \smallbreak
		On considère ($E$, $N$) un e.v.n. Soit $(u_n)_{n \in \mathbb{N}}$ une suite d'éléments de $E$. On dit que $u_n$ converge vers $\ell$ ssi la suite réelle ($N(u_n$ $-$ $\ell))_{n \in \mathbb{N}}$ converge vers $0$.
	\end{minipage}
}

\smallbreak
\underline{Propriété :} Supposons que $u_n$ $\tendplusinf$ $\ell$,  $u'_n$ $\tendplusinf$ $\ell'$. Alors,  $u_n$  $+$ $u'_n$ $\tendplusinf$ $\ell$ $+$ $\ell'$.

\smallbreak
\underline{Démo :} 
\parindent=1cm
\smallbreak
$N((u_n$ $+$ $u'_n)$ $-$ $(\ell$ $+$ $\ell'))$ $=$ $N((u_n$ $-$ $\ell)$ $+$ $(u'_n$ $-$ $\ell'))$ $\leqslant$ $\underset{\tendplusinf \text{ }0}{\underbrace{N(u_n\text{ }-\text{ }\ell)}}$ $+$ $\underset{\tendplusinf \text{ }0}{\underbrace{ N(u'_n\text{ }-\text{ }\ell')}}$ 
\smallbreak
donc $N((u_n$ $+$ $u'_n)$ $-$ $(\ell$ $+$ $\ell'))$ $\tendplusinf$ $0$. On en déduit que $u_n$  $+$ $u'_n$ $\tendplusinf$ $\ell$ $+$ $\ell'$
\smallbreak
\parindent=0cm
\underline{Remarque :} \parindent=1cm \smallbreak
Si $(u_n)_{n \in \mathbb{N}}$ une suite de ($E$, $N$) converge, la limite est unique : Supposons que $u_n$ $\longrightarrow$ $\ell$,  $u'_n$ $\longrightarrow$ $\ell'$. Alors, \smallbreak $0$ $\leqslant$ $N(\ell$ $-$ $\ell')$ $=$ $N((\ell$ $-$ $u_n)$ $+$ $(\ell'$ $-$ $u_n))$ $\leqslant$ $N(\ell$ $-$ $u_n)$ $+$ $N(\ell'$ $-$ $u_n)$ $\tendplusinf$ $0$\smallbreak $\Longrightarrow$ $0$ $\leqslant$ $N(\ell$ $-$ $\ell')$ $\leqslant$ \liminfty $\overset{n}{\underset{i = 1}{\sum}}u_i$ $=$ $0$ d'où $N(\ell$ $-$ $\ell')$ $=$ $0$, donc $\ell$ $-$ $\ell'$ $=$ $0$.

\smallbreak
\parindent=0cm
\underline{Proposition :} \parindent=1cm \smallbreak
Soient $N_1$, $N_2$ deux normes equivalentes sur un e.v $E$. Soit $(u_n)_{n \in \mathbb{N}}$ une suite de $E$, soit $\ell$ $\in$ $E$. Alors \smallbreak ($N_1(u_n$ $-$ $\ell)$ $\Longrightarrow$ $0$) $\Longleftrightarrow$ ($N_2(u_n$ $-$ $\ell)$ $\Longrightarrow$ $0$). (($i$) $\Longleftrightarrow$ ($ii$)).

\parindent=0cm
\underline{Démo :} \parindent=1cm \smallbreak Montrons que $i$/ $\Longrightarrow$ $ii$/. Comme $N_1$, $N_2$ sont équivalents, $\exists$ $C>0$, tel que $\forall$ $x$ $\in$ $E$, $N_2(x)$ $\leqslant$ $CN_1(x)$ donc \smallbreak $0$ $\leqslant$ $N_2(u_n$ $-$ $\ell)$ $\leqslant$ $\underset{ \tendplusinf \text{ }0}{\underbrace{CN_1(u_n\text{ }-\text{ }\ell)}}$. Donc $N_2(u_n$ $-$ $\ell)$ $\longrightarrow$ $0$. De même $ii$/ $\Longrightarrow$ $i$/ en inversant $N_1$ et $N_2$.

\parindent=0cm
\underline{Corollaire :} \parindent=1cm \smallbreak
Comme les normes ||..||$_1$, ||..||$_2$, ||..||$_{\infty}$ sont équivalentes : \smallbreak Une suite de $\mathbb{R}^p$ converge vers $\ell$ $\in$ $\mathbb{R}^p$ pour l'une de ces normes ssi elle converge pour une autre.

\parindent=0cm
\underline{Remarque :} \parindent=1cm \smallbreak
Pour montrer qu'une suite $(u_n)_{n \in \mathbb{N}}$ de ($E$, $N$) converge vers $\ell$, il est équivalent de montrer que \smallbreak la suite $(u_n$ $-$ $\ell)_{n \in \mathbb{N}}$ converge vers 0.

\parindent=0cm
\underline{Proposition :} \parindent=1cm \smallbreak
Soit $(u_n)_{n \in \mathbb{N}}$ une suite de $\mathbb{R}^p$, $u_n$ $=$ $
\begin{bmatrix}
	x_{1,n} \\
	\vdots \\
	x_{p,n}
\end{bmatrix}
$ où  ($(x_{j,n})_{n \in \mathbb{N}}$ est une suite de $\mathbb{R}$ $\forall$ $j \in \left\{ 1,\text{ }...,\text{ }p \right\} $). Alors \smallbreak $(u_n)_{n \in \mathbb{N}}$ converge vers $\ell$ $=$ $
\begin{bmatrix}
	\ell_1 \\
	\vdots \\
	\ell_p
\end{bmatrix}
$ $\in \mathbb{R^p}$ muni de l'une des normes ||..||$_1$, ||..||$_2$, ||..||$_{\infty}$ $\Longleftrightarrow$ Les suites $(x_{p,n})_{n \in \mathbb{N}}$ \smallbreak vers $\ell_j$, $\forall$ $j \in \left\{ 1,\text{ }...,\text{ }p \right\} $).

\parindent=0cm
\smallbreak
\underline{Démo :} \parindent=1cm \smallbreak Par la remarque précédente, on peut supposer $\ell$ $=$ $0$. On doit montrer ||$u_n$||$_k$ $\tendplusinf$ $0$ $\Longleftrightarrow$ \smallbreak $\forall$ $j \in \left\{ 1,\text{ }...,\text{ }p \right\} $  $(x_{j,n})_{n \in \mathbb{N}}$ vers $0$ (où $k$ $=$ $1$, $2$ ou $\infty$). Montrons $\Longrightarrow$. Supposons ||$u_n$||$_{\infty}$ $\tendplusinf$  $0$. Or, \smallbreak ||$u_n$||$_{\infty}$ $=$ Max[|$x_{1,n}$|, |$x_{2,n}$|, ..., |$x_{p,n}$|]. Alors |$x_{j,n}$| $\leqslant$ ||$u_n$||$_{\infty}$ $\tendplusinf$ $0$. Donc, $x_{j,n}$ $\tendplusinf$ $0$,\smallbreak $\forall$ $j \in \left\{ 1,\text{ }...,\text{ }p \right\} $.
\smallbreak
Montrons $\Longleftarrow$, On a : $x_{j,n}$ $\tendplusinf$ $0$,  $\forall$ $j \in \left\{ 1,\text{ }...,\text{ }p \right\} $. \smallbreak Alors ||$u_n$||$_1$ $=$ |$x_{1,n}$| $+$ |$x_{2,n}$| $+$ ... $+$ |$x_{p,n}$| $\tendplusinf$ $0$ $+$ $0$ $+$ ... $+$ $0$ $=$ $0$. Donc ||$u_n$||$_1$ $\tendplusinf$ $0$.

\subsection{Applications Continues}
\parindent=0cm
\fbox{
	\begin{minipage}{0.9\textwidth}
		\underline{Définition :} \smallbreak
		Soient ($E$, $N$), ($E'$, $N'$) deux e.v.n. Soient $A \subset E$, $f$ : $A$ $\longrightarrow$ $E'$, $a$ $\in$ $A$. On dit que $f$ est continue en $a$ ssi $\forall$ $\varepsilon >0$, $\exists$ $\eta$ $>0$ et $\forall$ $x$ $\in$ $A$, $\underset{x \in B_{E}(a,\text{ }\eta)}{\underbrace{N(x - a)< \eta}}$ $\Longrightarrow$ $\underset{f(x) \in B_{E'}(f(a),\text{ }\varepsilon)} {\underbrace{N'(f(x) - f(a)) < \varepsilon}}$.
	\end{minipage}
}


\smallbreak
\underline{Remarque :} \parindent=1cm \smallbreak
$\Longrightarrow$ $\forall$ $\varepsilon >0$, $\exists$ $\eta >0$ et $\forall$ $x$ $\in$ $A\cap B_E(a,\text{ }\eta)$, on a $f(x)$ $\in$ $B_{E'}(f(a),\text{ }\varepsilon)$
\medbreak
\begin{tikzpicture}[scale=0.75]
    \draw (0,0) ..controls +(6.5,0) and +(4,0).. (0,3);
    \draw (3.5,2) node{\footnotesize{$\bullet$}};
    \draw (3.5,2) node[below left]{\small{$\eta$}};
    \draw (3.5,2) node[above]{\small{$a$}};
    \draw (10.5,2.75) node[]{\small{$>$}};
    \draw [<->](3.5,2) -- (3.5,1.25);
    \draw (0.5,1.5) node{\Large{$A$}};
    \draw (3.5,2) circle (0.75);
    \draw (18,1.5)[pattern=north west lines] circle (0.75);
    \draw (18,1.5) circle (2);
    \draw (18,1.5) node{\footnotesize{$\bullet$}};
    \draw (18,0.75) node[below left]{\small{$\varepsilon$}};
    \draw (18,1.5) node[above]{\small{$f(a)$}};
    \draw [<->](18,1.5) -- (18,-0.5);
    \draw (3.5,2) ..controls +(1,1) and +(-1,1).. (17.5,2);
\end{tikzpicture}
\smallbreak
\parindent=0cm
\smallbreak
\underline{Remarque :} \parindent=1cm \smallbreak
Cela généralise la définition des fonctions continues d'une variable : $I$ intervalle de $\mathbb{R}$, $a$ $\in$ $I$, $f$ est continue \smallbreak en $a$ : $\forall$ $\varepsilon >0$, $\exists$ $\eta$ $>0$ et $\forall$ $x$ $\in$ $I$, $x \in$ |$x$ $-$ $a$| $<\eta$ $\Longrightarrow$ |$f(x)$ $-$ $f(a)$| $< \varepsilon$.

\parindent=0cm
\smallbreak
\underline{Proprosition :} \parindent=1cm \smallbreak
Soit $f$ : $A$ $\longrightarrow$ $E'$ continue en $a$ $\in$ $E$ . Alors, pour toute suite $(x_n)_{n \in \mathbb{N}}$ de $A$ convergeant vers $a$, la suite \smallbreak $f((x_n))_{n \in \mathbb{N}}$ converge vers $f(a)$ dans $E'$. 


\parindent=0cm
\smallbreak
\underline{Démo :} \parindent=1cm \smallbreak
Soit $\varepsilon>0$. Comme $f$ est continue en $a$, $\exists$  $\eta$ $>0$ tel que $\forall$ $x$ $\in$ $A$, $N(x$ $-$ $a)$ $<\eta$ $\Longrightarrow$ $N'(f(x)$ $-$ $f(a))$ $< \varepsilon$.
\smallbreak 
Soit ($x_n$) convergeant vers $a$, $x_n$ $\in$ $A$. Cela signifie que $N(x_n$ $-$ $a)$ $\tendplusinf$ $0$. Il existe donc $n_0$, tel que \smallbreak$\forall$ $n$ $\geqslant$ $n_0$, $N'(f(x)$ $-$ $f(a))$ $< \varepsilon$. On a prouvé que $\forall$ $\varepsilon$ $>0$, $\exists$  $n_0$, et $\forall$ $n$ $\geqslant$ $n_0$, $N'(f(x)$ $-$ $f(a))$ $< \varepsilon$.
\smallbreak Donc, $N'(f(x_n)$ $-$ $f(a))$ $\longrightarrow$ $0$ si $n\longrightarrow +\infty$ donc $f(x_n)$ $\longrightarrow$ $f(a)$ dans $E'$.

\parindent=0cm
\fbox{
	\begin{minipage}{0.9\textwidth}
		\underline{Théorème :} \smallbreak
		Soient ($E$, $N$), ($E'$, $N'$) deux e.v.n. Soient $A \subset E$, $f$ : $A$ $\longrightarrow$ $E'$, $a$ $\in$ $A$. \smallbreak $f$ est continue en $a$ $\Longleftrightarrow$ Pour toute suite $(x_n)_{n \in \mathbb{N}}$ de $A$ convergeant vers $a$, la suite $f(x_n)_{n \in \mathbb{N}}$ converge vers $f(a)$.
	\end{minipage}
}

\parindent=0cm
\smallbreak
\underline{Démo :} \parindent=1cm \smallbreak
$i$/ $\longrightarrow$ $ii$/ est la proposition précédente.
\smallbreak
On peut montrer $ii$/ $\longrightarrow$ $i$/ par contraposée. On suppose donc : $\exists$ $\varepsilon_0 >0$ et $\forall$ $\eta$ $>0$, $\exists$ $x$ $\in$ $A$ avec \smallbreak $N(x - a)< \eta$ et $N'(f(x) - f(a)) \geqslant \varepsilon_0$.
\smallbreak
Appliquons cela avec $\eta$ $=$ $\frac{1}{n+1}$ $(n \in \mathbb{N})$. Il existe donc pour tout $n \in \mathbb{N}$, $x_n$ $\in$ $A$ vérifiant $N(x_n - a)< \frac{1}{n+1}$ et \smallbreak $N'(f(x) - f(a)) \geqslant \varepsilon_0$.

\smallbreak
On a donc $N(x_n - a)$ $\tendplusinf$ $0$, donc $(x_n)_{n \in \mathbb{N}}$ est une suite de $A$ convergeant vers $a$. De plus la suite \smallbreak $f(x_n)_{n \in \mathbb{N}}$ ne peut converger vers $f(a)$ (puisque si elle convergeait vers $f(a)$, $0$ $=$ \liminfty $N'(f(x) - f(a)) \geqslant \varepsilon_0$ \smallbreak $>0$ : absurde.) On a montré que $i/$ (faux) $\Longrightarrow$ $ii/$ (faux)

\parindent=0cm
\smallbreak
\underline{Application du théorème précédent :}
\smallbreak
\underline{Proposition :} \parindent=1cm \smallbreak
Soient $E$, $E'$ deux e.v, $N$, $N_1$ deux normes équivalentes sur $E$, et $N'$, $N'_1$ deux normes équivalentes sur $E'$.
\smallbreak
Soient $A$ $\subset$ $E$, $a$ $\in$ $A$, $f$ : $A$ $\longrightarrow$ $E'$. Il y a équivalence entre :
\smallbreak $-$ $i/$ $f$ est continue en $a$ lorsque $E$ est muni de $N$, et $E'$ est muni de $N'$.
\smallbreak $-$ $ii/$ $f$ est continue en $a$ lorsque $E$ est muni de $N_1$, et $E'$ est muni de $N'_1$.
\parindent=0cm
\smallbreak
\underline{Démo :} \parindent=1cm \smallbreak
Par le théorème précédent, $i/$ équivaut à :
\smallbreak $-$ $i'/$ $\forall$ $(x_n)_{n \in \mathbb{N}}$ de $A$ vérifiant $N(x_n - a)$ $\longrightarrow $ $0$, on a : $N'(f(x) - f(a))$ $\longrightarrow $ $0$. \smallbreak De même $ii/$ équivaut à :
\smallbreak $-$ $ii'/$ $\forall$ $(x_n)_{n \in \mathbb{N}}$ de $A$ vérifiant $N_1(x_n - a)$ $\longrightarrow $ $0$, on a : $N_1'(f(x) - f(a))$ $\longrightarrow $ $0$.
\smallbreak Or, on a vu que si $N$ est équivalente à $N_1$: ($N(x_n - a)$ $\longrightarrow $ $0$) $\Longleftrightarrow$ ($N_1(x_n - a)$ $\longrightarrow $ $0$). De même, comme \smallbreak $N'$ est équivalente à $N'_1$, ($N'(f(x) - f(a))$ $\longrightarrow $ $0$) $\Longleftrightarrow$ ($N_1'(f(x) - f(a))$ $\longrightarrow $ $0$). Donc $i'/$ $\Longleftrightarrow$ $ii'/$.

\parindent=0cm
\smallbreak
\underline{Remarque :} \parindent=1cm \smallbreak
Supposons $E$ $=$ $\mathbb{R}^n$, $E'$ $=$ $\mathbb{R}^p$. On sait que les normes ||..||$_1$, ||..||$_2$, ||..||$_{\infty}$, sont équivalentes. Lorsqu'on \smallbreak étudie la continuité de $f$ : $A$ $\longrightarrow$ $E'$. où $A$ $\subset$ $E$, on peut étudier n'importe laquelle de ces normes.

\parindent=0cm
\subsection{Sommes de fonctions continues en un point}
\underline{Notation :} \parindent=1cm \smallbreak $f$ : $A$ $\longrightarrow$ $E'$, $g$ : $A$ $\longrightarrow$ $E'$, on note $f$ $+$ $g$ : $A$ $\longrightarrow$ $E'$, $x$ $\mapsto$ $(f$ $+$ $g)(x)$ $=$ $f(x)$ $+$ $g(x)$. Si $\lambda$ $\in$ $\mathbb{R}$, on pose \smallbreak $(\lambda \cdot f)(x)$ $:=$ $\lambda f(x)$, $\forall$ $x$ $\in$ $a$.
\parindent=0cm
\smallbreak
\underline{Proposition :} \parindent=1cm \smallbreak
Si $f$ et $g$ sont continues en $a$ $\in$ $A$ alors $f$ $+$ $g$ et $\lambda f(x)$ sont continues en $a$.

\parindent=0cm
\smallbreak
\underline{Démo :} \parindent=1cm \smallbreak Pour voir que $f$ $+$ $g$ est continue en $a$, il suffit de montrer que $\forall$ $(x_n)_{n \in \mathbb{N}}$ de $A$ convergeant vers $a$, \smallbreak $((f$ $+$ $g)(x_n))_{n \in \mathbb{N}}$ converge vers $(f$ $+$ $g)(a)$. Or comme $f$ est continue en $a$, $f(x_n)$ $\longrightarrow$ $f(a)$ \smallbreak et $g$ est continue en $a$, $g(x_n)$ $\longrightarrow$ $g(a)$. Donc $f(x_n)$ $+$ $g(x_n)$ $\longrightarrow$ $f(a)$ $+$ $g(a)$.

\parindent=0cm
\smallbreak
\underline{Proposition :} \parindent=1cm \smallbreak
Soient ($E$, $N$), ($E'$, $N'$), ($E''$, $N''$) trois e.v.n. Soient $A$ $\subset$ $E$, $B$ $\subset$ $E'$, 
$f$ : $A$ $\longrightarrow$ $E'$, $g$ : $B$ $\longrightarrow$ $E''$. \smallbreak Supposons $f(A)$ $\subset$ $B$. On peut donc définir $g$ $\circ$ $f$ : $A$ $\longrightarrow$ $E''$. Soit $a$ $\in$ $A$. On pose $b$ $=$ $f(a)$ ($\in$ $f(A)$ $\subset$ $B$) \smallbreak
Supposons $f$ continue en $a$ et $g$ continue en $b$. Alors $g$ $\circ$ $f$ continue en $g$ $\circ$ $f$ est continue en $a$.

\parindent=0cm
\smallbreak
\underline{Démo :} \parindent=1cm \smallbreak
Il suffit de voir que pour toute quite $(x_n)_{n \in \mathbb{N}}$ de $A$ convergeant vers $a$ $((g$ $\circ$ $f)(x_n))_{n \in \mathbb{N}}$  converge vers \smallbreak $(g$ $\circ$ $f)(a)$. Or comme $f$ est continue en $a$, $x_n$ $\longrightarrow$ $a$ $\Longleftrightarrow$ $a$, $f(x_n)$ $\longrightarrow$ $b$, et comme $g$ est continue en $b$, \smallbreak $y_n = f(x_n)$ $\longrightarrow$ $b$ $\Longleftrightarrow$ $g(y_n) =$ $(g$ $\circ$ $f)(x_n)$ $\longrightarrow$ $g(b)$ $=$ $(g$ $\circ$ $f)(a)$

\parindent=0cm
\smallbreak
\underline{Proposition :} \parindent=1cm \smallbreak
Soient $(E$, $N)$ un e.v.n. $A$ $\subset$ $E$, $a$ $\in$ $E$, $f$ : $A$ $\longrightarrow$ $\mathbb{R}^p$. On munit $\mathbb{R}^p$ de l'une des normes ||..||$_1$, ||..||$_2$, ||..||$_{\infty}$. \smallbreak Pour $x$ $\in$ $A$, écrivons $f(x)$ $=$ $
\begin{bmatrix}
	f_1(x) \\
	\vdots \\
	f_p(x)
\end{bmatrix}
$ $\in$ $\mathbb{R}^p$ On obtient $f_j$ : $A$ $\longrightarrow$ $\mathbb{R}$, $j \in \left\{ 1,\text{ }...,\text{ }p \right\} $. Il y a équivalence entre :
\smallbreak 
$-$ $i/$ $f$ est continue en $a$. \smallbreak
$-$ $ii/$ $\forall$ $j \in \left\{ 1,\text{ }...,\text{ }p \right\} $, $f_j$ est continue en $a$.

\parindent=0cm
\smallbreak
\underline{Démo :} \parindent=1cm \smallbreak
$-$ $i/$ $\Longleftrightarrow$ 
$(x_n)_{n \in \mathbb{N}}$ de $A$ $\longrightarrow$ $a$, la suite $f((x_n))_{n \in \mathbb{N}}$ converge vers $f(a)$ dans $\mathbb{R}$.
\smallbreak
$-$ $ii/$ $\Longleftrightarrow$ 
$(x_n)_{n \in \mathbb{N}}$ de $A$ $\longrightarrow$ $a$, $\forall$ $j \in \left\{ 1,\text{ }...,\text{ }p \right\} $ la suite $f_j((x_n))_{n \in \mathbb{N}}$ $\longrightarrow$ $f_j(a)$.
\smallbreak
Or on a vu que $
\begin{bmatrix}
	f_1(x_n) \\
	\vdots \\
	f_p(x_n)
\end{bmatrix}
$ $\longrightarrow$ 
$
\begin{bmatrix}
	f_1(a) \\
	\vdots \\
	f_p(a)
\end{bmatrix}
$
$\Longleftrightarrow$ $\forall$ $j \in \left\{ 1,\text{ }...,\text{ }p \right\}$ $f_j((x_n))_{n \in \mathbb{N}}$ $\tendplusinf$ $f_j(a)$.

\parindent=0cm
\fbox{
	\begin{minipage}{0.9\textwidth}
		\underline{Définition :} \smallbreak
		On dit que $f$ : $A$ $\longrightarrow$ $E'$ est continue sur $A$ $\Longleftrightarrow$ $\forall$ $a$, $\in$ $A$ $f$ est continue est $a$.
	\end{minipage}
}
\smallbreak
\parindent=0cm
\fbox{
	\begin{minipage}{0.9\textwidth}
		\underline{Théorème :} \smallbreak
		Soient ($E$, $N$), ($E'$, $N'$) deux e.v.n. Soient $u$ : $E$ $\longrightarrow$ $E'$ linéaire. Il y a équivalence entre : \smallbreak
		$-$ $i/$ $u$ est continue sur $E$ \smallbreak
		$-$ $ii/$ $u$ est continue sur $0$ \smallbreak
		$-$ $iii/$ $\exists$ $C>0$, $\forall$ $x$ $\in$ $E$, $N'(u(x))$ $\leqslant$ $CN(u(x))$.
	\end{minipage}
}
\parindent=0cm
\smallbreak
\underline{Démo :} \parindent=1cm \smallbreak
$i/$ $\Longrightarrow$ $ii/$ : évident. $0$ $\in$ $E$ \smallbreak
$ii/$ $\Longrightarrow$ $iii/$ : Si $u$ est continue en $0$, $\forall$ $\varepsilon >0$ et $\exists$ $\eta$ $>0$ et $\forall$ $x$ $\in$ $E$, $N(x$ $-$ $0)$ $<$ $\eta$ $\Longrightarrow$ $N'(u(x)$ $-$ $u(0))$ $<$ $\varepsilon$
\smallbreak
Comme $u$ est linéaire, $u(0)$ $=$ $0$ donc $N(x)$ $<$ $\eta$ $\Longrightarrow$ $N'(u(x))$ $<$ $\varepsilon$
Appliquons cela avec $\varepsilon$ $=$ $1$. \smallbreak $\exists$ $\eta_0$ $>0$ tel que $N(x)$ $<\eta_0$ $\Longrightarrow$ $N'(u(x))$ $<$ $1$. Soit $y$ $\in$ $E$, $y$
$\neq$ $0$. Posons $x$ $=$ $\frac{y}{N(y)}\cdot \frac{\eta_{\footnotesize{0}}}{2}$. Alors $N(x)$ $=$  $\frac{\eta_0}{2}$ \smallbreak $<$ $\eta_0$ donc soit $N'(\frac{\eta_0}{2N(y)}u(y))$ $<1$ puisque $u$ est linéaire. Donc, $\forall$ $y$ $\in$ $E \smallsetminus \left\{ 0 \right\}$ $\frac{N'(u(y))}{2N(y)/ \eta_0}$ $<1$ d'où \smallbreak $N'(u(y))$ $<$ $CN(y)$ avec $C = \frac{2}{\eta_0}$
Donc $iii/$ est vraie.
\smallbreak
$iii/$ $\Longrightarrow$ $i/$ : Soit $a$ $\in$ $E$. On veut monter que $u$ est continue en $a$. $\exists$ $C>0$ et $\forall$ $y$ $\in$ $E$, $N'(y)$ $<$ $CN(y)$.
\smallbreak
Pour $\varepsilon$ $>0$ donné, posons $\eta$ $=$ $\varepsilon/C$, supposons $N(x$ $-$ $a)$ $<\eta$. Alors, \smallbreak $N'(u(x)$ $-$ $u(a))$ $=$ $N'(u(x$ $-$ $a))$ $\leqslant$ $CN(x$ $-$ $a)$ $<$ $C\eta$ $=$ $\varepsilon$. Donc $u$ est continue en $a$.

\parindent=0cm
\smallbreak
\underline{Proposition :} \parindent=1cm \smallbreak
Munissons $\mathbb{R}^p$ de l'une des normes ||..||$_1$, ||..||$_2$, ||..||$_{\infty}$. Soient $(E$, $N)$ un e.v.n et $u$ : $\mathbb{R}^p$ $\longrightarrow$ $E$, une \smallbreak application linéaire. Alors $u$ est continue.
\parindent=0cm
\smallbreak
\underline{Démo :} \parindent=1cm \smallbreak
Notons ($e_1$, ..., $e_p$) la base canonique de $\mathbb{R}^p$, si $x$ $\in$ $\mathbb{R}^p$, $\overset{p}{\underset{j = 1}{\sum}}x_j e_j$. Alors $u(x)$ $=$ $\overset{p}{\underset{j = 1}{\sum}}x_j u(e_j)$ car $u$ est linéaire. \smallbreak Alors $N(u(x))$ $=$ $N( \overset{p}{\underset{j = 1}{\sum}}x_j u(e_j))$ $\leqslant$ 
$ \overset{p}{\underset{j = 1}{\sum}}N(x_j u(e_j))$ $=$ $ \overset{p}{\underset{j = 1}{\sum}}|x_j|N(u(e_j))$. Posons $M$ $=$ Max[$N'(u(e_j))$]. Donc \smallbreak $N(u(x))$ $\leqslant$ $M \overset{p}{\underset{j = 1}{\sum}}|x_j|$ $=$ $M||x||_1$. D'après $iii/$ de la propriété pécédente, cela implique que $u$ est continue.

\subsection{Exemples de fonctions continues}
\parindent=0cm
\smallbreak
\underline{Ex 1 :} \parindent=1cm \smallbreak
L'application $E$ $\times$ $E$ $\longrightarrow$ $E$, $(x$, $y)$ $\mapsto$ $x$ $+$ $y$ est continue.

\parindent=0cm
\smallbreak
\underline{Ex 2 :} \parindent=1cm \smallbreak
L'application $\mathbb{R}$ $\times$ $\mathbb{R}$ $\longrightarrow$ $\mathbb{R}$, ($x$, $y$) $\mapsto$ $x$ $\cdot$ $y$ est continue.

\parindent=0cm
\smallbreak
\underline{Ex 3 :} \parindent=1cm \smallbreak
L'application $\mathbb{C}$ $\times$ $\mathbb{C}$ $\longrightarrow$ $\mathbb{C}$, ($z$, $\omega$) $\mapsto$ $z$ $\cdot$ $\omega$ est continue.

\parindent=0cm
\smallbreak
\underline{Ex 4 :} \parindent=1cm \smallbreak
Soient $(E$, $N)$ un e.v.n, $A$ $\subset$ $E$, $f$ : $A$ $\longrightarrow$ $\mathbb{R}$*. Soit $a$ $\in$ $E$, supposons $f$ continue en $a$. Alors, $x$ $\mapsto$ $\frac{1}{f(x)}$ \smallbreak est continue en $a$.

\parindent=0cm
\smallbreak
\underline{Ex 5 :} \parindent=1cm \smallbreak
Soit $E$ $=$ $M_p(\mathbb{R})$ l'ensemble des matrices carrées d'ordre $p$. Si $A$ $=$ $(a_{ij})_{1 \leqslant i,j \leqslant p}$ $\in$ $M_p(\mathbb{R})$, posons \smallbreak $N(A)$ $=$ $p\underset{1 \leqslant i,j \leqslant p}{\text{Max}}$ |$a_{ij}$|. $N$ est une norme et on pose $N'$ une norme sur $E$, telle que $\forall$ $A$, $B$ $\in$ $M_p(\mathbb{R}$ \smallbreak $N'((A$, $B))$ $=$ Max[$N(A)$, $N(B)$]. Soit $\Phi$ : $E$ $\times$ $E$ $\longrightarrow$ $E$. ($A$, $B$) $\longrightarrow$ $AB$. 























\end{document}